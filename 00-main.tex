\PassOptionsToPackage{colorlinks=true,linkcolor=blue,urlcolor=blue}{hyperref}



\documentclass[11pt]{article}

%\usepackage[top=1in, bottom=1in, left=1cm, right=1cm]{geometry}
%\usepackage{lipsum}
%\usepackage{background}
%\backgroundsetup{
%  position=current page.east,
%  angle=-90,
%  nodeanchor=east,
%  vshift=-5mm,
%  opacity=1,
%  scale=3,
%  contents=Confidential
%}

\usepackage{amsmath,amssymb,amsthm,bm}   % TeX4ht relies on these
\usepackage{mathtools}            % optional but harmless
\usepackage{rotating}

\usepackage[margin=1cm, paperwidth=8.5in, paperheight=11in]{geometry}
\usepackage[
  hypertexnames=false
]{hyperref}
\usepackage{kotex}
\usepackage{etex}
\usepackage{fullpage}
\usepackage{mathtools}
\usepackage{amsfonts}
\usepackage{graphicx}
\usepackage{amsthm}
\usepackage[utf8]{inputenc}
\usepackage{afterpage}
\usepackage{adjustbox}
\usepackage{placeins}
\usepackage{fixltx2e}
\usepackage{supertabular} 
\usepackage{titlesec}
\usepackage{array}
\usepackage{tabularx}
\usepackage{color}
\usepackage{algorithm}
\usepackage{algpseudocode}
\usepackage{subcaption}
\usepackage{hyperref}
\usepackage{pgfplots}
\usepackage{relsize}
\let\labelindent\relax
\usepackage{enumitem}
\usepackage{fancyhdr}
\usepackage{alltt}
\usepackage{soul}
\usepackage{fancyvrb}
\usepackage{xcolor}
%\usepackage[english,greek]{babel}
%\usepackage{ucs} 
%\usepackage[utf8x]{inputenc}
%\usepackage[usenames,dvipsnames]{xcolor}
%\usepackage{tikz}
%\usepackage{tkz-tab}
%\usepackage{caption}
%\usepackage{latexsym}
%\usepackage{amssymb}
\usepackage[margin=0.1cm]{subcaption}
\usepackage{multicol}

\usepackage{breakurl} % Break URL
\usepackage{multirow}
\usepackage[most]{tcolorbox}
%\usepackage[breakable,skins]{tcolorbox}

\tcbset{breakable}
\usepackage{esvect}
\usepackage{hyperref}
\usepackage{mathdots}
\usepackage{pifont} 
\usepackage{booktabs}
\usepackage{arydshln}
\usepackage{xcolor}
\def\UrlBreaks{\do\/\do-}
\usepackage{mathtools}  
\usepackage{array}
\usepackage{hyperref}

\usepackage{hyperref}
\usepackage{mathtools, nccmath}
%\usepackage[style=numeric,sorting=none]{biblatex}

\usepackage{titling}
\renewcommand\maketitlehooka{\null\mbox{}\vfill}
\renewcommand\maketitlehookd{\vfill\null}

\DeclarePairedDelimiter{\nint}\lfloor\rceil

\DeclarePairedDelimiter{\abs}\lvert\rvert
\usepackage{tocloft}
\usepackage{pdflscape}
\usepackage[all=normal, floats, bibnotes, wordspacing, charwidths, indent]{savetrees}
\makeatletter%
\captionsetup{belowskip=0pt}

\usepackage{enumitem}
\setlist{nolistsep}
%\setlist[itemize]{leftmargin=0.1}
\setlist[itemize]{leftmargin=*}
\setlist[enumerate]{leftmargin=*}

% align figures to the top margin
\makeatletter
\setlength{\@fptop}{0pt}

\newcommand\bunderline{}% check for being undefined
\DeclareRobustCommand\bunderline[1]{\mathord{\mathpalette\b@underline{#1}}}
\newcommand{\b@underline}[2]{%
  \sbox\z@{$\m@th#1\underline{#2}$}%
  \raisebox{-\dp\z@}{\scalebox{.5}[.25]{$\m@th#1[$}}%
  \copy\z@
  \raisebox{-\dp\z@}{\scalebox{.5}[.25]{$\m@th#1]$}}%
}

\makeatother

%\renewcommand{\figureautorefname}{foo}
%\renewcommand{\tableautorefname}{bar}

%\usepackage[all=normal, floats, bibnotes, wordspacing, charwidths, indent, lists]

%\captionsetup{belowskip=0pt}

\usepackage{graphicx}
\def\rotatecharone#1{\rotatebox[origin=c]{180}{#1}}

\def\rotatechartwo#1{\reflectbox{#1}}
\usepackage{hyperref}
\skip\footins 0.5\baselineskip %8.1pt plus 4pt minus 2pt
\floatsep 0.5\baselineskip
\textfloatsep 0.5\baselineskip
\intextsep 0.5\baselineskip 
\dbltextfloatsep 0.5\baselineskip  
\dblfloatsep 0.5\baselineskip 

%\raggedbottom

\belowcaptionskip 0pt

\everypar{\looseness=-1 }

\setlength{\abovecaptionskip}{0pt} % 

\newenvironment{myproof}[1][\proofname]{%
  \begin{proof}[#1]$ $\par\nobreak\ignorespaces
}{%
  \end{proof}
}
\makeatletter
\renewcommand{\sectionautorefname}{\S\@gobble}
\renewcommand{\subsectionautorefname}{\S\@gobble}
\renewcommand{\subsubsectionautorefname}{\S\@gobble}
\renewcommand{\appendixautorefname}{\S\@gobble}
\renewcommand{\appendixautorefname}{\S\@gobble}
\makeatother
\newcommand{\ignore}[1]{}

% Add additional comma in the math mode
%\DeclareMathSymbol{,}{\mathpunct}{letters}{"2C}
%\newcommand{\comma}{\mathpunct{\mkern\thinmuskip}}
%\mathcode`,=\string"8000
%{\catcode`,=\active \gdef,{\comma}}


\addtolength{\cftsecnumwidth}{15pt}
\addtolength{\cftsubsecnumwidth}{10pt}
\addtolength{\cftsubsubsecnumwidth}{10pt}

\makeatletter%
\usepackage{stackengine,xcolor}
\input pdf-trans
\newbox\qbox
\def\usecolor#1{\csname\string\color@#1\endcsname\space}
\newcommand\bordercolor[1]{\colsplit{1}{#1}}
\newcommand\fillcolor[1]{\colsplit{0}{#1}}
\newcommand\colsplit[2]{\colorlet{tmpcolor}{#2}\edef\tmp{\usecolor{tmpcolor}}%  
  \def\tmpB{}\expandafter\colsplithelp\tmp\relax%
  \ifnum0=#1\relax\edef\fillcol{\tmpB}\else\edef\bordercol{\tmpC}\fi}
\def\colsplithelp#1#2 #3\relax{%
  \edef\tmpB{\tmpB#1#2 }%
  \ifnum `#1>`9\relax\def\tmpC{#3}\else\colsplithelp#3\relax\fi
}
\newcommand\outline[1]{\leavevmode%
  \def\maltext{#1}%
  \setbox\qbox=\hbox{\maltext}%
  \boxgs{Q q 2 Tr \thickness\space w \fillcol\space \bordercol\space}{}%
  \copy\qbox%
}
\newcommand\mathcalbb[2][1]{%
  \stackengine{0pt}{\def\thickness{.55}\outline{$\mathcal{#2}$}}{\kern.1pt\outline{$\mathcal{#2}$}}{O}{l}{F}{F}{L}}
\bordercolor{black}
\fillcolor{white}
\def\thickness{.1}% TO CHANGE THICKNESS OF SHADOW
\usepackage{lmodern}
\usepackage{lipsum}


\makeatletter
\DeclareRobustCommand{\ccong}{\mathrel{\mathpalette\@verequiv\sim}}
\newcommand{\@verequiv}[2]{%
  \lower.5\p@\vbox{
    \lineskiplimit\maxdimen
    \lineskip-.5\p@
    \ialign{%
      $\m@th#1\hfil##\hfil$\crcr
      #2\crcr
      \equiv\crcr
    }%
  }%
}
\renewcommand*\env@matrix[1][c]{\hskip -\arraycolsep
  \let\@ifnextchar\new@ifnextchar
  \array{*\c@MaxMatrixCols #1}}


\newcommand{\hathat}[1]{% 
\begingroup%
  \let\macc@kerna\z@%
  \let\macc@kernb\z@%
  \let\macc@nucleus\@empty%
  \hat{\raisebox{.2ex}{\vphantom{\ensuremath{#1}}}\smash{\hat{#1}}}%
\endgroup%
}

\makeatother

% \title{**\vspace*{\fill}**The Beginner's Textbook for Fully Homomorphic Encryption}

% \date{January 1, 2025}

% \author{Ronny Ko**\vspace*{\fill}**}


\begin{document}
\title{\Huge{\textbf{The Beginner's Textbook}}\\ \Huge{\textbf{for Fully Homomorphic Encryption}}}
\author{\textbf{Ronny Ko}\\{LG Electronics Inc.}}%\\\texttt{\small{hajoon.ko@lge.com}}}
\date{January 1, 2025}



\begin{titlingpage}
\maketitle
\end{titlingpage}


\clearpage

\begin{abstract}
Fully Homomorphic Encryption (FHE) is a cryptographic scheme that enables computations to be performed directly on encrypted data, as if the data were in plaintext. After all computations are performed on the encrypted data, it can be decrypted to reveal the result. The decrypted value matches the result that would have been obtained if the same computations were applied to the plaintext data.

FHE supports basic operations such as addition and multiplication on encrypted numbers. Using these fundamental operations, more complex computations can be constructed, including subtraction, division, logic gates (e.g., AND, OR, XOR, NAND, MUX), and even advanced mathematical functions such as ReLU, sigmoid, and trigonometric functions (e.g., sin, cos). These functions can be implemented either as exact formulas or as approximations, depending on the trade-off between computational efficiency and accuracy. 

Fully Homomorphic Encryption (FHE) enables privacy-preserving machine learning by allowing a server to process the client’s data in its encrypted form through an ML model. With FHE, the server learns neither the plaintext version of the input features nor the inference results. Only the client, using their secret key, can decrypt and access the results at the end of the service protocol.
FHE can also be applied to confidential blockchain services, ensuring that sensitive data in smart contracts remains encrypted and confidential while maintaining the transparency and integrity of the execution process.
Other applications of FHE include secure outsourcing of data analytics, encrypted database queries, privacy-preserving searches, efficient multi-party computation for digital signatures, and more.

This book is designed to help the reader understand how FHE works from the mathematical level. The book comprises the following four parts: 

$ $

\begin{itemize}
\item \textbf{\autoref{part:basic-math}:~\nameref{part:basic-math}} explains necessary background concepts for FHE, such as Group, Field, Order, Polynomial Ring, Cyclotomic Polynomial, Vectors and Matrices, Chinese Remainder Theorem, Taylor Series, Polynomial Interpolation, and Fast Fourier Transform.

\item \textbf{\autoref{part:pqc}:~\nameref{part:pqc}} explains well-known lattice-based cryptographic schemes, which are LWE, RLWE, GLWE, GLev, and GGSW cryptosystems.

\item \textbf{\textbf{\autoref{part:generic-fhe}:~\nameref{part:generic-fhe}}} explains the generic techniques of FHE adopted by many existing schemes, such as homomorphic addition, multiplication, modulus switching, and key switching. 


\item \textbf{\textbf{\autoref{part:fhe-schemes}:~\nameref{part:fhe-schemes}}} explains four widely used FHE schemes: TFHE, BFV, CKKS, and BGV, as well as their RNS-variant versions.
\end{itemize}

$ $

These parts are designed in an incremental manner, and therefore understanding each part requires the understanding of its prior part(s). 

$ $

Please report any bugs or suggestions regarding the draft to the \href{https://github.com/gogo9th/fhe-textbook/issues}{\textbf{GitHub Issues Board}}. 

\end{abstract}

\subsubsection*{Acknowledgments}
Special thanks are extended to Robin Geelen (KU Leuven) for his thoughtful and dedicated comments, and to Yongwoo Lee (Inha University) for his general advice.

\thispagestyle{empty}

\newpage

\tableofcontents



%\newcommand{\subparagraph}{} % /usepackage{titlesec}
\titleformat*{\section}{\LARGE\bfseries\scshape}
\titleformat*{\subsection}{\Large\bfseries}
\titleformat*{\subsubsection}{\bfseries}
\titleformat*{\paragraph}{\itshape\subsubsectionfont}
\titleformat*{\subparagraph}{\large\bfseries}

% page header foot 
%\usepackage{fancyhdr}
%\pagestyle{fancy}
%\lhead{Security and Privacy in Cyber-Physical Systems: Foundations and Applications}
%\rfoot{Copyright \textcopyright 2016 by Wiley}
% \thispagestyle{fancy}, after \maketitle
\newcommand{\para}[1]{\vspace{0.05in}\noindent{\bf{#1}}}


\newcommand{\white}[1]{{\textcolor{white}{#1}}} % phantom upgrade

\newcommand{\gap}[1]{\text{ } \text{#1} \text{ }}

\newcommand{\tboxlabel}[1]{$\bm{\langle}$Summary~{#1}$\bm{\rangle}$}

\newcommand{\tboxdef}[1]{$\bm{\langle}$Definition~{#1}$\bm{\rangle}$}

\newcommand{\tboxtheorem}[1]{$\bm{\langle}$Theorem~{#1}$\bm{\rangle}$}

\newtheorem{proposition}{Proposition}



\newtcolorbox[blend into=tables]{mytable}[2][]{float=htb, halign=center,  title={#2}, every float=\centering, #1}


% $\hat Y = \frac{1}{1 + e^{-Z}}$.

% $Z = {w_1 \cdot X_1 + w_2 \cdot X_2 + \dots + w_n \cdot X_n + b}$



\clearpage

%\section{Background}


\part{Basic Math}
\label{part:basic-math}

\renewcommand{\thesection}{A-\arabic{section}}
\setcounter{section}{0}

This chapter explains the basic mathematical components of the Number theory: group, field, order, roots of unity, cyclotomic polynomial, polynomial ring, and decomposition. These are essential building blocks for post-quantum cryptography.

\clearpage

\section{Modulo Arithmetic}
\label{sec:modulo}
\noindent \textbf{- Reference:} 
\href{https://www.youtube.com/watch?v=fz1vxq5ts5I}{YouTube -- Extended Euclidean Algorithm Tutorial}


\subsection{Overview}

\begin{tcolorbox}[title={\textbf{\tboxdef{\ref*{subsec:group-def}} Integer Modulo}}]

\begin{itemize}

\item \textbf{modulo} is an operation of computing the remainder after dividing a number by another number. \textbf{modulo} is often abbreviated as \textbf{mod}.

$ $

\item \textbf{$\bm{a}$ mod $\bm{q}$ (i.e., $\bm{a} \bm{\text{ modulo } q}$)} is the remainder after dividing $a$ by $q$, which is always one element among $\{0, 1, 2, 3, \cdots, q-1\}$. For example, $7 \bmod 5 = 2$, because the remainder of dividing 7 by 5 is 2. 

$ $

\item \textbf{modulus:} Given $\bm{a}$ mod $\bm{q}$, we call the divider $q$ the modulus, whereas modulo is an operation.

$ $

\item \textbf{Modulo Congruence ($\bm{\equiv}$):} $a$ is congruent with $b$ modulo $a$ (i.e., $a \equiv b \textbf{ mod } q$) if they have the same remainder when divided by $a$. For example, $5 \equiv 12 \bmod 7$, because $5 \bmod 7 = 5$ and $12 \bmod 7 = 5$. In mathematics, the notation $a \equiv b \bmod q$ is identical to $a = b \pmod q$, meaning that the remainder of $a$ divided by $q$ is the same as the remainder of $b$ divided by $q$. Note that this notation is different from $a = b \bmod q$, meaning that $a$ is identical to the remainder of $b$ divided by $q$, 

$ $

\item \textbf{Congruence \textit{v.s.} Equality:} 

$a \equiv b \bmod q \Longleftrightarrow a = b + k\cdot q$ \text{ } (for some integer $k$)

$ $

This means that $a$ and $b$ are congruent modulo $q$ if and only if $a$ and $b$ are different by some multiple of $q$. For example, $5 \equiv 12 \bmod 7 \Longleftrightarrow 5 = 12 + (-1)\cdot 7$

\end{itemize}

\end{tcolorbox}


\subsection{Modulo Arithmetic}
\label{subsec:modulo-arithmetic}


The supported modulo operations are addition, subtraction, and multiplication. The properties of these modulo operations are as follows:

\begin{tcolorbox}[title={\textbf{\tboxtheorem{\ref*{subsec:group-def}.1} Properties of Modulo Operations}}]
For any integer $x$, the following is true:

\begin{enumerate}
\item \textbf{Addition:} $a \equiv b \bmod q \Longleftrightarrow a + x\equiv b + x\bmod q$

\item \textbf{Subtraction:} $a \equiv b \bmod q \Longleftrightarrow a - x\equiv b - x\bmod q$

\item \textbf{Multiplication:} $a \equiv b \bmod q \Longleftrightarrow a \cdot x\equiv b \cdot x\bmod q$

\end{enumerate}

\end{tcolorbox}

\begin{proof}

$ $

For any integer $x$,

\begin{enumerate}


\item \textbf{Addition:} $a \equiv b \bmod q \Longleftrightarrow a = b + k q $ (for some $k$) \textcolor{red}{ \text{ } \# $a$ and $b$ differ by some multiple of $q$}

$\Longleftrightarrow a + x = b + k\cdot q + x$

$\Longleftrightarrow a + x = b + x + k\cdot q$ \textcolor{red}{ \text{ }  \# $a+x$ and $b+x$ differ by some multiple of $q$}

$\Longleftrightarrow a + x \equiv b + x \bmod q$

$ $

\item \textbf{Subtraction:} $a \equiv b \bmod q \Longleftrightarrow a = b + k q $ (for some $k$)

$\Longleftrightarrow a - x = b + k\cdot q - x$

$\Longleftrightarrow a - x = b - x + k\cdot q$  \textcolor{red}{ \text{ }  \# $a-x$ and $b-x$ differ by some multiple of $q$}

$\Longleftrightarrow a - x \equiv b - x \bmod q$

$ $

\item \textbf{Multiplication:} $a \equiv b \bmod q \Longleftrightarrow a = b + k q $ (for some $k$)

$\Longleftrightarrow a \cdot x = b \cdot x + k\cdot q \cdot x$

$\Longleftrightarrow a \cdot x = b \cdot x + k_x\cdot q$ (where $k_x = k \cdot x$)  \textcolor{red}{ \text{ }  \# $a\cdot x$ and $b\cdot x$ differ by some multiple of $q$}

$\Longleftrightarrow a \cdot x = b \cdot x \pmod q$

\end{enumerate}
\end{proof}

Based on the modulo operations in Theorem\ref*{subsec:group-def}.1, we can also derive the following properties of modulo arithmetic. 

\begin{tcolorbox}[title={\textbf{\tboxtheorem{\ref*{subsec:group-def}.2} Properties of Modulo Arithmetic}}]

\begin{enumerate}
\item \textbf{Associative:} $(a \cdot b) \cdot c \equiv a \cdot (b \cdot c) \bmod q$

\item \textbf{Commutative:} $(a \cdot b) \equiv (b \cdot a) \bmod q$

\item \textbf{Distributive:} $(a \cdot (b + c)) \equiv ((a \cdot b) + (a \cdot c))  \bmod q$

\item \textbf{Interchangeable:} Congruent values are interchangeable in the modulo arithmetic. 

For example, suppose $(a \equiv b \bmod q)$ and $(c \equiv d \bmod q)$. Then, $a$ and $c$ are interchangeable, and $b$ and $d$ are interchangeable in modulo arithmetic as follows:

$(a + c) \equiv (c + d) \equiv (a + d) \equiv (b + c) \bmod q$

$(a - c) \equiv (c - d) \equiv (a - d) \equiv (b - c) \bmod q$

$(a \cdot c) \equiv (c \cdot d) \equiv (a \cdot d) \equiv (b \cdot c) \bmod q$

\end{enumerate}

\end{tcolorbox}

The proof of Theorem~\ref*{subsec:group-def}.2 is similar to that of Theorem~\ref*{subsec:group-def}.1, which we leave as an exercise for the reader. 


\subsubsection{Inverse}
\label{subsec:modulo-inverse}

\begin{tcolorbox}[title={\textbf{\tboxdef{\ref*{subsec:modulo-inverse}} Inverse in Modulo Arithmetic}}]


In modulo $q$ (i.e., in the world of remainders where all numbers have been divided by $q$), for each $a \in \{0, 1, 2, \cdots, q-1\}$:

\begin{itemize}

\item \textbf{Additive Inverse} of $a$ is denoted as $a_+^{-1}$ that satisfies $a + a_+^{-1} \equiv 0 \bmod q$. For example, in modulo 11, $3_+^{-1} = 8$, because $3 + 8 \equiv 0 \bmod 11$.

\item \textbf{Multiplicative Inverse} of $a$ is denoted as $a_*^{-1}$ that satisfies $a + a_*^{-1} \equiv 1 \bmod q$. For example, in modulo 11, $3_*^{-1} = 4$, because $3 \cdot 4 \equiv 1 \bmod 11$.

\end{itemize}

\end{tcolorbox}

\subsubsection{Modulo Division}
\label{subsec:modulo-division}

In modulo arithmetic, \textit{modulo division} is different from regular numeric division. In fact, there is no such thing as \textit{modulo division}, because modulo is already a division operation that outputs a remainder. \textit{modulo division} of $b$ by $a \bmod q$ is equivalent to computing the modulo multiplication $b \cdot a_*^{-1} \bmod q$. The result of \textit{modulo division} is different from that of numeric division, because \textit{modulo division} always gives some integer (as it multiplies two integers modulo $q$), whereas numeric division gives a real number. The inverse of an integer modulo $q$ can be computed by the extended Euclidean algorithm (\href{https://www.youtube.com/watch?v=fz1vxq5ts5I}{YouTube tutorial})

\subsubsection{Centered Residue Representation}
\label{subsec:modulo-centered}

Throughout this section, we have assumed that the residues are positive integers. For example, the possible residues for $\bmod \text{ } q$ are assumed to be $\{0, 1, \cdots, q-1\}$. This system is called canonical (i.e., unsigned) residue representation. On the other hand, there is also a counterpart system that assumes signed (i.e., centered) residues $\left\{-\dfrac{q}{2}, -\dfrac{q}{2} + 1, \cdots, 0, \cdots, \dfrac{q}{2} - 2, \dfrac{q}{2} - 1\right\}$, where the residues are centered around $0$ and the number of total residues is the same: $q$. In both systems, a modulo operation changes a given value to another value within the system's residue range such that: (1) if the given value is bigger than the upper bound of the residue range, the value is subtracted by the modulus $q$; (2) if the value is smaller than the lower bound of the residue range, the value is added by the modulus $q$. The only difference between these two (canonical and centered) systems is their upper bounds and lower bounds: $0$ and $q-1$ in the canonical residue system, whereas $-\dfrac{q}{2}$ and $\dfrac{q}{2} - 1$ in the centered residue system. The canonical residue representation assumes that $\mathbb{Z}_q = \{0, 1, \cdots, q-1\}$, whereas the centered residue system assumes that $\mathbb{Z}_q = \left\{-\dfrac{q}{2}, -\dfrac{q}{2} + 1, \cdots, 0, \cdots, \dfrac{q}{2} - 2, \dfrac{q}{2} - 1\right\}$. 

In both systems, the same modulo property of addition, subtraction, multiplication, and division holds, which can be proved by applying the same reasoning described in \autoref{subsec:modulo-arithmetic}: the same properties hold in both systems because any two congruent residues in the centered system are separated by the $kq$ gaps (for some integer $k$) in both systems. 

Also, the same property holds for an inverse: an inverse of $a$ modulo $q$ is $a^{-1}$ such that $a \cdot a^{-1} \equiv 1 \bmod q$. 

Using a signed residue representation is useful in certain cases. In an example of canonical (i.e., unsigned) residue representation, suppose we have the relation $a + b \bmod q$ and we know that in a given application, $a + b$ is guaranteed to be within the $[0, q-1]$ range (i.e., $0 \leq a + b \leq q-1$). Then, $(a + b \bmod q)$ = $a + b$, and thus we can remove the modulo operation, simplifying the relation. Now, suppose a different example of centered (i.e., signed) residue representation where we have the relation $a - b \bmod q$, and we know that in a given application, $a - b$ is guaranteed to be within the range $\left[-\dfrac{q}{2}, \dfrac{q}{2} - 1\right]$. Then, $(a - b \bmod q) = a - b$. However, notice that if the relation $a - b \bmod q$ were in a canonical residue representation, then we cannot remove the modulo operation, because if $a - b$ is negative, then this becomes smaller than the lower bound of the canonical residue system (i.e., $0$), and thus a modulo reduction (i.e., addition by one or more $q$) is needed. 

In \autoref{subsec:rns-fastbconvex}, we design the \textsf{FastBConvEx} operation based on this beneficial property of centered residue representation: in this algorithm design, we can simplify $(\mu + u \bmod b_\alpha)$ to $\mu + u$, because we know that $\dfrac{b_\alpha}{2} \leq \mu + u < \dfrac{b_\alpha}{2}$.

\clearpage

\section{Group}
\label{sec:group}
%\textbf{- First Read:} 
%\href{https://e.math.cornell.edu/people/belk/numbertheory/CyclotomicPolynomials.pdf}{Fields and Cyclotomic Polynomials}

\subsection{Definitions}
\label{subsec:group-def}


\begin{tcolorbox}[title={\textbf{\tboxdef{\ref*{subsec:group-def}} Group}}]
\noindent \textbf{\underline{Set Elements}}
\begin{itemize}
\item \textbf{Set ($\mathbb{S}$):} A bundle of elements: $\mathbb{S} = \{a, b, c, \gap{$\cdots$}, n\}$
\item \textbf{Set Operations $\bm{(+, \cdot)}$:} A set defines two operations between any two elements $a, b \in \mathbb{S}$ as operands: addition $(+)$ and multiplication $(\cdot)$
\item \textbf{Additive Identity ($1_{(+)}$):} An element $i \in \mathbb{S}$ is an additive identity if for all $a \in \mathbb{S}$, $i + a = a$.
\item \textbf{Multiplicative Identity ($1_{(\cdot)}$):} An element $i \in \mathbb{S}$ is an additive identity if for all $a \in \mathbb{S}$, $i \cdot a = a$
\item \textbf{Additive Inverse ($a^{-1}$):} For each $a \in \mathbb{S}$, its additive inverse $a^{-1}$ is defined as an element such that $a + a^{-1} = 1_{(+)}$ (i.e., additive identity)
\item \textbf{Multiplicative Inverse ($a^{-1}$):} For each $a \in \mathbb{S}$ except for $a = 0$, its multiplicative inverse $a^{-1}$ is defined as an element such that $a \cdot a^{-1} = 1_{(\cdot)}$ (i.e., multiplicative identity)
\end{itemize}

$ $

\noindent \textbf{\underline{Element Operation Features}}
\begin{itemize}
\item \textbf{Closed:} A set $\mathbb{S}$ is closed under the $(+)$ operation if for every $a, b \in \mathbb{S}$, it is the case that $a + b \in \mathbb{S}$. Likewise, a set $\mathbb{S}$ is closed under the $(\cdot)$ operation if for every $a, b \in \mathbb{S}$, it is the case that $a \cdot b \in \mathbb{S}$. 
\item \textbf{Associative:} $(a + b) + c = a + (b + c)$
\item \textbf{Commutative:} $a + b = b + a$
\item \textbf{Distributive:} $a \cdot (b + c) = (a \cdot b) + (a \cdot c)$
\end{itemize}


$ $

\noindent \textbf{\underline{Group Types}}
\begin{itemize}
\item \textbf{Semigroup:} A semigroup is a set of elements which is closed and associative on a single operation ($+$ or $\cdot$)
\item \textbf{Monoid:} A monoid is a semigroup, plus it has an identity element, which returns the other operand over the set operation.

(e.g., $0$ is the identity element for $+$ operator, $1$ is the identity element for the $\cdot$ operator)
\item \textbf{Group:} A group is a monoid, plus every element has an inverse (except for that of 0, the identity)
\item \textbf{Abelian Group:} An abelian group is a group, plus its operation is commutative.
\end{itemize}
\end{tcolorbox}

\subsection{Examples}
\label{subsec:group-ex}

$\mathbb{Z}$ (i.e., the set of all integers) is an abelian group under addition ($+$), because:
\begin{itemize}
\item \textbf{Closed:} For any integer $a, b \in \mathbb{Z}$, $a + b = c$ is also an integer ($\in \mathbb{Z}$).
\item \textbf{Associative:} For any integer $a, b, c \in \mathbb{Z}$, $(a + b) + c = a + (b + c)$.
\item \textbf{Identity:} The additive identity is 0, because for any $a \in \mathbb{Z}$, $a + 0 = a$.
\item \textbf{Inverse:} For each $a \in \mathbb{Z}$, its additive inverse is $-a$, as $a + (-a) = 0$.
\item \textbf{Commutative: } For any integer $a, b \in \mathbb{Z}$, $a + b = b + a$.
\end{itemize}

$ $

\noindent $\mathbb{Z}$ is a monoid under multiplication ($\cdot$), because:
\begin{itemize}
\item \textbf{Closed:} For any integer $a, b \in \mathbb{Z}$, $a \cdot b = c$ is also an integer ($\in \mathbb{Z}$).
\item \textbf{Associative:} For any integer $a, b, c \in \mathbb{Z}$, $(a \cdot b) \cdot c = a \cdot (b \cdot c)$.
\item \textbf{Identity:} The multiplicative identity is 1, because for any $a \in \mathbb{Z}$, $a \cdot 1 = a$.
\item \textbf{NO Inverse:} For an integer $a \in \mathbb{Z}$, its multiplicative inverse is $\dfrac{1}{a}$, but this is not necessarily an integer ($\not\in \mathbb{Z}$), and thus breaks the closure property.
\end{itemize}
$ $

\noindent $\mathbb{R}$ (i.e., the set of all real numbers) is an abelian group under multiplication ($\cdot$), because: 
\begin{itemize}
\item \textbf{Closed:} For any real number $a, b \in \mathbb{R}$, $a \cdot b = c$ is also a real number ($\in \mathbb{R}$).
\item \textbf{Associative:} For any real number $a, b, c \in \mathbb{R}$, $(a \cdot b) \cdot c = a \cdot (b \cdot c)$.
\item \textbf{Identity:} The multiplicative identity is 1, as for any real number $a \in \mathbb{R}$, $a \cdot 1 = a$.
\item \textbf{Inverse:} For each real number $a \in \mathbb{R}$ (except for 0, the identity), its multiplicative inverse is $\dfrac{1}{a}$, which is a real number ($\in \mathbb{R}$).
\end{itemize}


\clearpage

\section{Field}
\label{sec:field}
\textbf{- Reference:} 
\href{https://e.math.cornell.edu/people/belk/numbertheory/CyclotomicPolynomials.pdf}{Fields and Cyclotomic Polynomials}~\cite{cyclotomic-polynomial}

\subsection{Definitions}
\label{subsec:field-def}

\begin{tcolorbox}[title={\textbf{\tboxdef{\ref*{subsec:field-def}} Field Definitions}}]
\begin{itemize}
\item \textbf{Ring:} A set of elements which is an abelian group under the + operator, and closed, associative, and distributive on the ($+, \cdot$) operators. 
\item \textbf{Field:} A set of elements which is an abelian group under both the $(+, \cdot)$ operators (i.e., the set has an identity element and multiplicative inverses for all elements), and distributive on those operators.
\item \textbf{Galois Field ($\text{GF}(p^n)$):} A field with a finite number of elements (whose number must be $p^n$ for some prime $p$ and a positive integer $n$).
\item \textbf{$\mathbb{Z}_p$ ($\mathbb{Z}/p\mathbb{Z}$):} The finite field of integer modulo $p$, which is $\{0, 1, 2, \cdots... \text{ } p - 1\}$ where $p$ is a prime number. If $p$ is a prime number, $\mathbb{Z}_p$ is always a finite field. This is also called a quotient ring of $p$.
\end{itemize}
\end{tcolorbox}
$ $

\subsection{Examples}
\label{subsec:field-ex}

$\mathbb{Z}$ (the set of all integers) is a ring, but not a field, because not all of its elements have a multiplicative inverse (as shown in \autoref{subsec:group-ex}). 

$ $

\noindent $\mathbb{R}$ (the set of all real numbers) is a field. As shown in \autoref{subsec:group-ex}, it is an abelian group over the $(+)$ and $(\cdot)$ operators, and its elements are distributive over the $(+, \cdot)$ operators.


$ $

\noindent $\mathbb{Z}_7 = \{0, 1, 2, 3, 4, 5, 6\}$ is a finite field because:
\begin{itemize}
\item \textbf{Closed:} For any $a, b \in \mathbb{Z}_7$, there exists some $c_1 \in \mathbb{Z}_7$ such that $a + b \equiv c_1 \bmod 7$ and, some $c_2 \in \mathbb{Z}_7$ such that $a \cdot b \equiv c_2 \bmod 7$.
\item \textbf{Associative:} For any $a, b, c \in \mathbb{Z}_7$, $ (a + b) + c = a + (b + c)$ and $(a \cdot b) \cdot c = a \cdot (b \cdot c)$.
\item \textbf{Commutative:} For any $a, b \in \mathbb{Z}_7$, $ a + b = b + a$, and $a \cdot b = b \cdot a$.
\item \textbf{Distributive:} For any $a, b, c \in \mathbb{Z}_7$, $ (a + b) \cdot c = a \cdot c + b \cdot c$.
\item \textbf{Identity:} For any $a \in \mathbb{Z}_7$, its additive identity is $0$, and its multiplicative identity is $1$.
\item \textbf{Inverse:} For any $a \in \mathbb{Z}_7$, there exists an additive inverse $a_1' \in \mathbb{Z}_7$ such that $a_1 + a_1' \equiv 0 \bmod 7$. For example, if $a_1 = 3$, then its additive inverse $a_1' = 4$, because $3 + 4 = 7 \equiv 0 \bmod 7$. Also, for any $a_2 \in \mathbb{Z}_7$ (except for 0), there exists a multiplicative inverse $a_2' \in \mathbb{Z}_7$ such that $a_2 \cdot a_2' \equiv 1 \bmod 7$. For example, if $a_2 = 3$, then its multiplicative inverse $a_2' = 5$, because $3 \cdot 5 = 15 \equiv 1 \bmod 7$. 
\end{itemize}

\subsection{Theorems}
\label{subsec:field-theorem}

\begin{tcolorbox}[title={\textbf{\tboxtheorem{\ref*{subsec:field-theorem}} Field Theorems}}]
\begin{enumerate}
\item \textbf{Size of Finite Field:} A finite field is called Galois field and always has $p^n$ elements (where $p$ is a prime and $n$ is a positive integer).
\item \textbf{Isomorphic Fields:} Any two finite fields, $\mathbb{F}_1$ and $\mathbb{F}_2$ with the same number of elements are isomorphic (i.e., there exists a bi-jective one-to-one mapping function $f : \mathbb{F}_1 \rightarrow \mathbb{F}_2$ and the algebraic operations $(+, \cdot)$ preserve correctness among newly mapped elements). In other words, there exists a mapping function $f : \mathbb{F}_1 \rightarrow \mathbb{F}_2$ comprised of the field operators ($+$, $\cdot$). For such an isomorphic function $f$, for any $a, b \in \mathbb{F}_1$, $f(a+b) = f(a) + f(b)$ and $f(a \cdot b) = f(a) \cdot f(b)$
\end{enumerate}
\end{tcolorbox}


\clearpage

\section{Order}
\label{sec:order}
\textbf{- Reference:} 
\href{https://e.math.cornell.edu/people/belk/numbertheory/CyclotomicPolynomials.pdf}{Fields and Cyclotomic Polynomials}~\cite{cyclotomic-polynomial}

\subsection{Definitions}
\label{subsec:order-def}

\begin{tcolorbox}[title={\textbf{\tboxdef{\ref*{subsec:order-def}} Order Definition}}]
$\bm{\textsf{ord}_{\mathbb{F}}(a)}$: For $a \in \mathbb{F}$ (a finite field, \autoref{subsec:field-def}), $a$'s order is the smallest positive integer $k$ such that $a^k = 1$. 

$ $


\end{tcolorbox}

Note that the multiplicative group generated by $a$ as a generator excludes $\{0\}$, the identity, because $0^k = 0$ for all $k$ values.

\subsection{Theorems}
\label{subsec:order-theorem}



\begin{tcolorbox}[title={\textbf{\tboxtheorem{\ref*{subsec:order-theorem}.1} Order Property (I)}}]
For $a \in \mathbb{F}$, and $n \geq 1$, $a^n = 1$ if and only if \textbf{\textsf{ord}}$_{\mathbb{F}}(a) \text{ } | \text{ } n$ 

(i.e., $\textsf{ord}_{\mathbb{F}}(a)$ divides $n$).
\end{tcolorbox}

\begin{myproof}
    \begin{enumerate}
    \item \textit{Forward Proof:} If $\textsf{ord}_{\mathbb{F}}(a) \text{ } | \text{ } n$, then for $\textsf{ord}_{\mathbb{F}}(a) = k$ where $k$ is $a$'s order, and $n = lk$ for some integer $l$. 
    
    Then, $a^n = a^{lk} = (a^k)^l = 1^l = 1$.
    \item \textit{Backward Proof:} If $a^n = 1$, then for $\textsf{ord}_{\mathbb{F}}(a) = k$, $n \geq k$, because by definition of order, $k$ is the smallest value that satisfies $a^k = 1$. For any $n > k$, $n$ has to be a multiple of $k$, because in order to satisfy $a^n = a^k \cdot a^{n - k} = 1 \cdot a^{n - k} = 1$, the next smallest possible value for $n$ is $2k$ (as $a^k$ is the smallest value that is equal to 1). By induction, the possible values of $n$ are $n = k, 2k, 3k, \cdots$.
    \end{enumerate}
\end{myproof}


\begin{tcolorbox}[title={\textbf{\tboxtheorem{\ref*{subsec:order-theorem}.2} Order Property (II)}}]
If $\textsf{ord}_{\mathbb{F}}(a) = k$, then for any $n \geq 1$, $\textsf{ord}_{\mathbb{F}}(a^n) = \dfrac{k} {\text{gcd}(k, n)}$.
\end{tcolorbox}
\begin{myproof}
    \begin{enumerate}
    \item $a^k, a^{2k}, a^{3k}, ... \text{ } = 1$. 
    \item Given $\textsf{ord}_{\mathbb{F}}(a^n) = m$, $(a^n)^m, (a^n)^{2m}, (a^n)^{3m}, ... \text{ } = 1$ 
    \item Note that by definition of order, $x=k$ is the smallest value that satisfies $a^x$ = 1. Thus, given $\textsf{ord}_{\mathbb{F}}(a^n) = m$, then $m$ is the smallest integer that makes $(a^n)^m = 1$. Note that $(a^n)^m$ should be a multiple of $a^k$, which means $mn$ should a multiple of $k$. The smallest possible integer $m$ that makes $mn$ a multiple of $k$ is $m = \dfrac{k}{\text{gcd}(k, n)}$. 
    \end{enumerate}
\end{myproof}

\begin{tcolorbox}[title={\textbf{\tboxtheorem{\ref*{subsec:order-theorem}.3} Order Property (III)}}]
Given $k \text{ } | \text{ } n$, $\textsf{ord}_{\mathbb{F}}(a) = kn$ if and only if $\textsf{ord}_{\mathbb{F}}(a^k) = n$.
\end{tcolorbox}
\begin{myproof}
\begin{enumerate}
    \item \textit{Forward Proof:} Given $\textsf{ord}_{\mathbb{F}}(a) = kn$, and given Theorem~\ref*{subsec:order-theorem}.2, $\textsf{ord}_{\mathbb{F}}(a^k) = \dfrac{nk}{\text{gcd}(k, nk)} = \dfrac{nk}{k} = n$.
    \item \textit{Backward Proof:} Given $\textsf{ord}_{\mathbb{F}}(a^k) = n$, we should prove that $\textsf{ord}_{\mathbb{F}}(a) = nk$. Let $\textsf{ord}_{\mathbb{F}}(a) = m$. Then, by Theorem\ref*{subsec:order-theorem}.2, $\textsf{ord}_{\mathbb{F}}(a^k) = \dfrac{m}{\text{gcd}(m, k)}$, which is $n$. In other words, $m = n \cdot \text{gcd}(m, k)$. But as $k \text{ } | \text{ } n$, it is also true that $k \text{ } | \text{ } m$. Then, $\text{gcd}(m, k) = k$. Therefore, $m = n \cdot \text{gcd}(m, k) = nk$.
\end{enumerate}
\end{myproof}

\begin{tcolorbox}[title={\textbf{\tboxtheorem{\ref*{subsec:order-theorem}.4} Fermat's Little Theorem}}]
Given $|\mathbb{F}| = p$ (a prime) and $a \in \mathbb{F}$, $a^p = a$.
\end{tcolorbox}
\begin{myproof}
    \begin{enumerate}
    \item $\{a^1, a^2, ... \text{ } a^p\}$ forms a multiplicative subgroup $H$ of the group $G = \mathbb{F^{\times}}$ (i.e., $\mathbb{F}$ without $\{0\}$). 
    \item Lagrange's Group Theory states that for any subgroup $H$ in a group $G$, $|H|$ divides $|G|$. This means that given $\textsf{ord}_G(a) = k$ (where $a^k = 1$), $k$ divides $|G|$ (where $|G| = p - 1$). 
    \item Therefore, given $kl = p - 1$ for some integer $l$, $a^{p-1} = (a^k)^l = 1^l = 1$. Thus, $a^p = a$.
    \end{enumerate}
\end{myproof}



\clearpage

\section{Polynomial Ring}
\label{sec:polynomial-ring}
\input{a05-polynomial-ring}

\clearpage

\section{Decomposition}
\label{sec:decomp}
Decomposition is a mathematical technique to convert a large-base number into a mathematical formula expressing the same value in a smaller base. This section will explain number decomposition and polynomial decomposition. 

\subsection{Number Decomposition}
\label{subsec:number-decomp}
Suppose we have the number $\gamma$ modulo $q$. Number decomposition expresses $\gamma$ as a sum of multiple numbers in base $\beta$ as follows: 

$ $

$\gamma = \gamma_1 \dfrac{q}{\beta^1} + \gamma_2 \dfrac{q}{\beta^2} + \cdots + \gamma_l \dfrac{q}{\beta^l}  $

$ $

\noindent , where each of the $\gamma_i$ term represents a base-$\beta$ value shifted $i$ bits (which is a multiple of $\text{log}_2 \beta$) to the left. We call $l$ the level of decomposition. This is visually shown in \autoref{fig:decomp}. %We will cover the GLev ciphertext formula in \autoref{sec:glev}.

\begin{figure}[h!]
    \centering
  \includegraphics[width=0.8\linewidth]{figures/decomp.pdf}
  \caption{An illustration of number decomposition.}
  \label{fig:decomp}
\end{figure}

We define the decomposition of number $\gamma$ into base $\beta$ with level $l$ as follows:


$ $

$\textsf{Decomp}^{\beta, l}(\gamma) = (\gamma_1, \gamma_2, \text{ } \cdots , \text{ } \gamma_l)$.
 
$ $

Number decomposition is also called radix decomposition, where the base $\beta$ is called a radix. 


\subsubsection{Example}

Suppose we have the number $\gamma = 13 \bmod q$, where $q = 16$. Suppose we want to decompose 13 with the base $\beta = 2$ and level $l = 4$. Then, 13 is decomposed as follows:

$ $

$13 = 1 \cdot \dfrac{16}{2^1} + 1 \cdot \dfrac{16}{2^2} + 0 \cdot \dfrac{16}{2^3} + 1 \cdot \dfrac{16}{2^4}$

$ $

$\textsf{Decomp}^{2, 4}(13) = (1, 1, 0, 1)$


\subsection{Polynomial Decomposition}
\label{subsec:poly-decomp}

This time, suppose we have polynomial $f$ in the polynomial ring ${\mathbb{Z}_q[x] / (x^n + 1)}$. Therefore, each coefficient $c_i$ of $f$ is an integer modulo $q$. Polynomial decomposition expresses $f$ as a sum of multiple polynomials in base $\beta$ and level $l$ as follows:


\begin{tcolorbox}[title={\textbf{\tboxlabel{\ref*{subsec:poly-decomp}} Polynomial Decomposition}}]

Given $f \in \mathbb{Z}_q[z]/(x^n+1)$, where:

$ $

$f = f_1 \dfrac{q}{\beta^1} + f _2\dfrac{q}{\beta^2} + \cdots + f_l \dfrac{q}{\beta^l}  $

$ $

We denote the decomposition of polynomial $f$ into base $\beta$ with level $l$ as follows:

$ $

$\textsf{Decomp}^{\beta, l}(f) = (f_1, f_2, \text{ } \cdots , \text{ } f_l)$
 $ $
\end{tcolorbox}




\subsubsection{Example}

Suppose we have the following polynomial in the polynomial ring $\mathbb{Z}_{16}[x] / (x^4 + 1)$:

$ $

$f = 6x^3 + 3x^2 + 14x + 7 \in \mathbb{Z}_{16}[x] / (x^4 + 1)$

$ $

Suppose we want to decompose the above polynomial with base $\beta = 4$ and level $l = 2$. Then, each degree's coefficient of the polynomial $f$ is decomposed as follows:

$ $

${\bm{x}^{\bm{3}}}$: $6 = \color{blue}{1 \cdot \dfrac{16}{4^1}} \color{black}+ \color{red}{2 \cdot \dfrac{16}{4^2}}$

${\bm{x}^{\bm{2}}}$: $3 = \color{blue}{0 \cdot \dfrac{16}{4^1}} \color{black}+ \color{red}{3 \cdot \dfrac{16}{4^2}}$

${\bm{x}^{\bm{1}}}$: $14 = \color{blue}{3 \cdot \dfrac{16}{4^1}} \color{black}+ \color{red}{2 \cdot \dfrac{16}{4^2}}$

${\bm{x}^{\bm{0}}}$: $7 = \color{blue}{1 \cdot \dfrac{16}{4^1}} \color{black}+ \color{red}{3 \cdot \dfrac{16}{4^2}}$

$ $

The decomposed polynomial is as follows:

$f = 6x^3 + 3x^2 + 14x + 7 = \color{blue}{(1x^3 + 0x^2 + 3x + 1) \cdot \dfrac{16}{4^1}} \color{black}+ \color{red}{(2x^3 + 3x^2 + 2x + 3) \cdot \dfrac{16}{4^2}} \color{black}$

$ $

$\textsf{Decomp}^{4, 2}(6x^3 + 3x^2 + 14x + 7) = (1x^3 + 0x^2 + 3x + 1, 2x^3 + 3x^2 + 2x + 3)$

\subsubsection{Discussion}

Note that after decomposition, the original coefficients of the polynomial got reduced to smaller numbers. This characteristic is importantly used in homomorphic encryption's multiplication, for reducing the growth rate of the noise. Normally, the polynomial coefficients of ciphertexts are large because they are uniformly random numbers. Reducing such big constants is important to reduce the noise growth during homomorphic multiplication. We will discuss this more in detail in \autoref{subsec:tfhe-mult-cipher}.


\subsection{Approximate Decomposition}
\label{subsec:approx-decomp}

\begin{figure}[h!]
    \centering
  \includegraphics[width=0.7\linewidth]{figures/decomp3.pdf}
  \caption{An illustration of approximate decomposition}
  \label{fig:decomp3}
\end{figure}

If the base $\beta$ does not divide the modulo $q$, then there exists no level $l$ such that $\beta^l \text{ } | \text{ } q$, thus some lower bits of $q$ have to be drawn away during decomposition, as shown in \autoref{fig:decomp3}. Such lower bits can be rounded to the nearest multiple of $\dfrac{q}{\beta^l}$ during decomposition. In such a case, the decomposition is an approximate decomposition. 




\subsection{Gadget Decomposition}
\label{subsec:gadget-decomposition}

Gadget decomposition is a generalized form of number decomposition (\autoref{subsec:number-decomp}). In number decomposition, a number $\gamma$ is decomposed as follows: 

$\gamma = \gamma_1 \dfrac{q}{\beta^1} + \gamma_2 \dfrac{q}{\beta^2} + \cdots + \gamma_l \dfrac{q}{\beta^l} $

$ $

In gadget decomposition, we decompose $\gamma$ as follows: 

$\gamma = \gamma_1 g_1 + \gamma_2 g_2 + \cdots + \gamma_l g_l $

$ $

We denote $\vec{g} = (g_1, g_2, \cdots, g_l)$ as a gadget vector, and $\textsf{Decomp}^{\vec{g}}(\gamma) = (\gamma_1, \gamma_2, \text{ } \cdots , \text{ } \gamma_l) $

$ $

Then, $\gamma = \langle \textsf{Decomp}^{\vec{g}}(\gamma), \vec{g} \rangle $

$ $

In the case of number decomposition (\autoref{subsec:number-decomp}), its gadget vector is $\vec{g} = \Bigg(\dfrac{q}{\beta}, \dfrac{q}{\beta^2}, \cdots, \dfrac{q}{\beta^l}\Bigg)$.

%After ciphertext-to-plaintext multiplication $\Lambda \cdot \textsf{GLWE}_{S, \sigma}(M)$, the noise grows from $E$ to $E' = \Lambda \cdot E$. To limit this noise growth, we introduce a technique called gadget decomposition, which consists of decomposed (\autoref{subsec:number-decomp}) $\Lambda$ and an GLev encryption (\autoref{subsec:glev-enc}) of $M$ as follows:

%$\Lambda = \Lambda_1 \dfrac{q}{\beta^1} + \Lambda_2 \dfrac{q}{\beta^2} + \cdots + \Lambda_l \dfrac{q}{\beta^l} \longrightarrow \textsf{Decomp}^{\beta, l}(\Lambda) = (\Lambda_1, \Lambda_2, \cdots, \Lambda_l)$

%$ $

%$\textsf{GLev}_{S, \sigma}^{\beta, l}(\Delta M) = \Bigg\{ \textsf{GLWE}_{S, \sigma}\left(\Delta M \dfrac{q}{\beta^1}\right), \textsf{GLWE}_{S, \sigma}\left(\Delta M \dfrac{q}{\beta^2}\right), \cdots \textsf{GLWE}_{S, \sigma}\left(\Delta M \dfrac{q}{\beta^l}\right) \Bigg\}$

%$ $

%Based on decomposed $\Lambda$ and the GLev encryption of $M$, we can compute the ciphertext-to-plaintext multiplication of $M \cdot \Lambda$ as follows:

%$\langle \textsf{Decomp}^{\beta, l}(\Lambda), \textsf{GLev}_{S, \sigma}^{\beta, l}(\Delta M) \rangle$


%$= \Lambda_1 \cdot \textsf{GLWE}_{S, \sigma}\left(\Delta M \dfrac{q}{\beta^1}\right) +  \Lambda_2 \cdot \textsf{GLWE}_{S, \sigma}\left(\Delta M \dfrac{q}{\beta^2}\right) \cdots \Lambda_l \cdot \textsf{GLWE}_{S, \sigma}\left(\Delta M \dfrac{q}{\beta^l}\right)$

%$= \textsf{GLWE}_{S, \sigma}\left(\Delta M \cdot \Lambda_1\dfrac{q}{\beta^1}\right) + \textsf{GLWE}_{S, \sigma}\left(\Delta M  \cdot \Lambda_2\dfrac{q}{\beta^2}\right) \cdots \cdot \textsf{GLWE}_{S, \sigma}\left(\Delta M  \cdot \Lambda_l\dfrac{q}{\beta^l}\right)$

%$= \textsf{GLWE}_{S, \sigma}\left(\Delta M  \cdot \left(  \Lambda_1 \dfrac{q}{\beta^1} + \Lambda_2 \dfrac{q}{\beta^2} + \cdots + \Lambda_l \dfrac{q}{\beta^l} \right) \right)$

%$= \textsf{GLWE}_{S, \sigma}(\Delta (M \cdot \Lambda))$

%$ $

%Given the noise of each GLWE ciphertext in the GLev ciphertext above is $E_i$, the aggregate noise of the final ciphertext-to-plaintext multiplication is $\sum\limits_{i=0}^{l}E_i$, which is much smaller than the original noise $\Lambda \cdot e$ in \tboxlabel{\ref*{subsec:tfhe-mult-plain}.3}-- as $l$ is usually much smaller than $\Lambda$. This way, we can limit the noise growth of ciphertext-to-plaintext multiplication. 

%We can do gadget decomposition for any number of consecutive ciphertext-to-plaintext multiplication operations, by postponing the summation of GLWE ciphertexts to the end. However, gadget decomposition gets terminated if there is any interruption of a different operation, such as ciphertext-to-plaintext addition, ciphertext-to-ciphertext addition, or ciphertext-to-ciphertext multiplication, because in that case, the consistent format of the $GLev$ ciphertext breaks. 

%There is a more general and powerful solution to re-initialize the accumulated noise $E$ without decrypting the ciphertext, which is called bootstrapping. We will later explain the TFHE scheme's bootstrapping in \autoref{subsec:tfhe-bootstrapping}). 




\bibliographystyle{unsrt}
\bibliography{z-bibfile}



%\clearpage
%\input{scratch-real}

\end{document}


%https://www.youtube.com/watch?v=vYKdh5oQ4Zw
