
In \autoref{sec:roots} and \autoref{sec:cyclotomic}, we learned about the definition and properties of the $\mu$-th roots of unity and the $\mu$-th cyclotomic polynomial over complex numbers (i.e., $X \in \mathbb{C}$) as follows: 

\begin{itemize}
\item \textbf{The $\bm \mu$-th roots of unity} are the solutions for $X^\mu = 1$ over $X \in \mathbb{C}$ (complex numbers). The formula for the $\mu$-th root of unity is $X = e^{2 \pi i k / \mu}$ for all integer $k$ such that $0 \leq k \leq \mu - 1$. 
\item \textbf{The primitive $\bm \mu$-th roots of unity (denoted as $\bm \omega$)} are those $\mu$-th roots of unity whose order (\autoref{subsec:order-def}) is $\mu$ (i.e., $\omega^{\mu} = 1$ and $\omega^{\frac{\mu}{2}} \neq 1$). 
\item Given any primitive $\mu$-th roots of unity $\omega$, it can generate all primitive $\mu$-th roots of unity by computing $\omega^{k'}$ such that $k'$ is an integer $0 < k' < \mu$ and $\textsf{gcd}(k', \mu) = 1$ (Theorem~\ref*{subsec:roots-theorem}.4 in \autoref{subsec:order-theorem}). 
\item \textbf{The $\bm \mu$-th cyclotomic polynomial} is defined as a polynomial whose roots are the primitive $\mu$-th roots of unity. That is, \[ \Phi_{\mu}(x) = \prod_{\omega \in P({\mu})} (x - \omega) = \prod_{\substack{0 \leq k \leq {\mu}-1,\\ \text{gcd}(k, {\mu}) = 1}} (x - \omega^k) \]
\end{itemize}

In this section, we will explain the $\mu$-th cyclotomic polynomial over $X \in \mathbb{Z}_p$ (integer ring), which is structured as follows: 

\begin{tcolorbox}[title={\textbf{\tboxdef{\ref*{sec:cyclotomic-polynomial-integer-ring}} Roots of Unity and Cyclotomic Polynomial over Integer Ring $\mathbb{Z}_p$}}]


\begin{itemize}
\item \textbf{The $\bm \mu$-th roots of unity (denoted as $\bm \omega$)} are the solutions for $X^\mu \equiv 1 \bmod p$. Note that these solutions are not $X = \omega^{2 \pi i k / \mu}$ (the formula for the solutions over $X \in \mathbb{C}$). 
\item \textbf{The primitive $\bm \mu$-th roots of unity} are defined as those $\mu$-th roots of unity whose order (\autoref{subsec:order-def}) is $\mu$ (i.e., $\omega^{\mu} \equiv 1 \bmod p$, and $\omega^{\lceil \frac{\mu}{2} \rfloor} \not\equiv 1 \bmod p$). 
\item Given any primitive $\mu$-th roots of unity $\omega$, it can generate all primitive $\mu$-th roots of unity by computing $\omega^{k'}$ such that $k'$ is an integer $0 < k' < \mu$ and $\textsf{gcd}(k', \mu) = 1$.
\item \textbf{The $\bm \mu$-th cyclotomic polynomial} is defined as a polynomial whose roots are the primitive $\mu$-th roots of unity. That is, \[ \Phi_{\mu}(x) = \prod_{\omega \in P({\mu})} (x - \omega) = \prod_{\substack{0 \leq k \leq {\mu}-1,\\ \text{gcd}(k, {\mu}) = 1}} (x - \omega^k) \]
\end{itemize}

\end{tcolorbox}

\begin{table}[h] %usepackage{array} 
\begin{tabular}{|>{\centering\arraybackslash}p{0.1\columnwidth}||>{\centering\arraybackslash}p{0.4\columnwidth}||>{\centering\arraybackslash}p{0.4\columnwidth}|}
\hline \hline
& \textbf{Polynomial over $\bm{X} \bm{\in} \bm{\mathbb{C}}$} & \textbf{Polynomial over $\bm{X} \in \bm{\mathbb{Z}}_{\bm{p}}$} \\ 
& \textbf{(Complex Number)} & \textbf{(Integer Ring)} \\ \hline \hline
\textbf{Definition of the $\bm \mu$-th Root of Unity}& All $X \in \mathbb{C}$ such that $X^\mu = 1$, (which are computed as $X = e^{2 \pi i k / \mu}$ for integer $k$ where $0 \leq k \leq \mu - 1$)& All $X \in \mathbb{Z}_p$ such that $X^\mu \equiv 1 \bmod p$\\ \hline
\textbf{Definition of the Primitive $\bm \mu$-th Root of Unity}& Those $\mu$-th roots of unity $\omega$ such that $\omega^{\mu} = 1$, and $\omega^{\frac{\mu}{2}} \neq 1$ &  Those $\mu$-th roots of unity $\omega$ such that $\omega^{\mu} \equiv 1 \bmod p$, and $\omega^{\frac{\mu}{2}} \not\equiv 1 \bmod p$ \\ \hline
\textbf{Definition of the $\bm \mu$-th Cyclotomic Polynomial} & \multicolumn{2}{|c|}{\shortstack{The polynomial whose roots are the $\mu$-th primitive roots of unity as follows: \\ $ \Phi_{\mu}(x) = \prod_{\omega \in P(\mu)} (x - \omega) $  \text{ } (see Definition~\ref*{subsec:cyclotomic-def} in \autoref{subsec:cyclotomic-def})}}\\ \hline
\textbf{Finding Primitive $\bm \mu$-th Roots of Unity} & For $\omega = e^{2 \pi i/ \mu}$, compute all $\omega^k$ such that $0 < k < \mu $ and $\textsf{gcd}(k, \mu) = 1$  (Theorem~\ref*{subsec:roots-theorem}.4 in \autoref{subsec:roots-theorem}) & Find one satisfactory $\omega$ that is a root of the $\mu$-th cyclotomic polynomial, and compute all $\omega^k \bmod p$ such that $0 < k < \mu $ and $\textsf{gcd}(k, \mu) = 1$ \\ \hline \hline
\end{tabular}
\caption{The roots of unity and cyclotomic polynomials over $X \in \mathbb{C}$ v.s. over $X \in \mathbb{Z}_p$}
\label{tab:cyclotomic-polynomial-comparison}
\end{table}

Note that in the $\mu$-th cyclotomic polynomial in both cases of over $X \in \mathbb{C}$ and over $X \in \mathbb{Z}_p$, each of their roots $\omega$ (i.e., the primitive $\mu$-th root of unity) has the order $\mu$ (i.e., $\omega^{\mu} = 1$ over $X \in \mathbb{C}$, and $\omega^{\mu} \equiv 1 \bmod p$ over $X \in \mathbb{Z}_p$). Also note that each root $\omega$ can generate all roots of the $\mu$-th cyclotomic polynomial by computing $\omega^{k'}$ such that $\textsf{gcd}(k', \mu) = 1$.

\autoref{tab:cyclotomic-polynomial-comparison} compares the properties of the roots of unity and the $\mu$-th cyclotomic polynomial over $X \in \mathbb{C}$ (complex numbers) and over $X \in \mathbb{Z}_p$ (integer ring).






\subsection{Vandermonde Matrix with Roots of a Cyclotomic Polynomial}
\label{subsec:vandermonde-euler-integer-ring}

Theorem~\ref*{subsec:vandermonde-euler} (in \autoref{subsec:vandermonde-euler}) showed that $V \cdot V^T = n \cdot I^R_n$, where $V$ is the Vandermonde matrix $V = \mathit{Vander}(x_0, x_1, \cdots, x_{n-1})$, where each $x_i$ is the primitive $\mu$-th root of unity over $X \in \mathbb{C}$, where $\mu$ is a power of 2. In this subsection, we will show that the relation $V \cdot V^T = n \cdot I^R_n$ holds even if each $x_i$ is the primitive $\mu$-th root of unity over $X \in \mathbb{Z}_p$. In particular, we will prove Theorem~\ref*{subsec:vandermonde-euler}: 


\begin{tcolorbox}[title={\textbf{\tboxtheorem{\ref*{subsec:vandermonde-euler}} Vandermonde Matrix with Roots of  \text{(power-of-2)}-th Cyclotomic Polynomial}}]

Suppose we have an $n \times n$ (where $n$ is a power of 2) Vandermonde matrix comprised of $n$ distinct roots of the $\mu$-th cyclotomic polynomial over $X \in \mathbb{Z}_p$ (integer ring), where $\mu$ is a power of 2 and $n = \dfrac{\mu}{2}$. In other words, $V = \mathit{Vander}(x_0, x_1, \cdots, x_{n-1})$, where each $x_i$ is the root of $X^n + 1$ (i.e., the primitive $(\mu=2n)$-th roots of unity). Then, the following holds:

$V \cdot V^T = \begin{bmatrix}
0 & \cdots & 0 & 0 & n\\
0 & \cdots & 0 & n & 0\\
0 & \cdots & n & 0 & 0\\
\vdots & \iddots & \vdots & \vdots & \vdots \\
n & 0 & 0 & \cdots & 0\\
\end{bmatrix} = n \cdot I^R_n$

$ $

And $V^{-1} = n^{-1}\cdot V^T \cdot I_n^R$


\end{tcolorbox}
\begin{proof}
$ $
\begin{enumerate}
\item $V \cdot V^T$ is expanded as follows:

$V \cdot V^T = \begin{bmatrix}
1 & (\omega) & (\omega)^2 & \cdots & (\omega)^{n-1}\\
1 & (\omega^3) & (\omega^3)^2 & \cdots & (\omega^3)^{n-1}\\
1 & (\omega^5) & (\omega^5)^2 & \cdots & (\omega^5)^{n-1}\\
\vdots & \vdots & \vdots & \ddots & \vdots \\
1 & (\omega^{2n-1}) & (\omega^{2n-1})^2 & \cdots & (\omega^{2n-1})^{n-1}\\
\end{bmatrix} 
\cdot 
\begin{bmatrix}
1 & 1 & 1 & \cdots & 1\\
(\omega) & (\omega^3) & (\omega^5) & \cdots & (\omega^{2n-1})\\
(\omega)^2 & (\omega^3)^2 & (\omega^5)^2 & \cdots & (\omega^{2n-1})^2\\
\vdots & \vdots & \vdots & \ddots & \vdots \\
(\omega)^{n-1} & (\omega^3)^{n-1} & (\omega^5)^{n-1} & \cdots & (\omega^{2n-1})^{n-1}\\
\end{bmatrix} $

$ $


$=
\begin{bmatrix}
\sum\limits_{k=0}^{n-1} \omega^{2k} & \sum\limits_{k=0}^{n-1} \omega^{4k}  & \sum\limits_{k=0}^{n-1} \omega^{6k} & \cdots & \sum\limits_{k=0}^{n-1} \omega^{2nk} \\

\sum\limits_{k=0}^{n-1} \omega^{4k} & \sum\limits_{k=0}^{n-1} \omega^{6k}  & \sum\limits_{k=0}^{n-1} \omega^{8k} & \cdots & \sum\limits_{k=0}^{n-1} \omega^{2k(n+1)} \\

\sum\limits_{k=0}^{n-1} \omega^{6k} & \sum\limits_{k=0}^{n-1} \omega^{8k} & \sum\limits_{k=0}^{n-1} \omega^{10k} & \cdots & \sum\limits_{k=0}^{n-1} \omega^{2k(n+2)} \\

\vdots & \vdots & \vdots & \ddots & \vdots \\
\sum\limits_{k=0}^{n-1} \omega^{2nk} & \sum\limits_{k=0}^{n-1} \omega^{2(n+1)k} & \sum\limits_{k=0}^{n-1} \omega^{2(n+2)k} & \cdots & \sum\limits_{k=0}^{n-1} \omega^{2(n+n-1)k} \\

\end{bmatrix}$

$ $

, where $\omega$ (i.e., the primitive $(\mu=2n)$-th root of unity) has the order $2n$. 

$ $

\item Note that the $V \cdot V^T$ matrix's anti-diagonal elements are $\sum\limits_{k=0}^{n-1} \omega^{2nk}$. It can be seen that $\omega^{2n} \equiv 1 \bmod p$, because $\textsf{Ord}_p(\omega) = 2n$. Thus, the $V \cdot V^T$ matrix's every anti-diagonal element is $\sum\limits_{k=0}^{n-1} 1 = n$.

$ $

\item Next, we will prove that the $V \cdot V^T$ matrix has $0$ for all other positions than the anti-diagonal ones. In other words, we will prove the following: 

$\sum\limits_{k=0}^{n-1} \omega^{2k} = \sum\limits_{k=0}^{n-1} \omega^{4k} = \sum\limits_{k=0}^{n-1} \omega^{6k} = \gap{$\cdots$} = \sum\limits_{k=0}^{n-1} \omega^{2(n-1)k} = \sum\limits_{k=0}^{n-1} \omega^{2(n+1)k} = \gap{$\cdots$} = \sum\limits_{k=0}^{n-1} \omega^{2(2n-1)k} = 0$

$ $

The above is true, because in the particular case of the $(\mu=2n)$-th cyclotomic polynomial $X^n + 1$ (where $n$ is a power of 2), $\omega^{i} + \omega^{i+\frac{n}{2}} \equiv 0 \bmod p$ for each integer $i$ where $0 \leq i \leq n - 1$. Therefore, in the element $\sum\limits_{k=0}^{n-1} \omega^{2k}$, its one-half terms add with its other-half terms and their final summation becomes $0$. This is the same for all other following elements: 

$\sum\limits_{k=0}^{n-1} \omega^{4k}, \text{ } \sum\limits_{k=0}^{n-1} \omega^{6k}, \text{ } \gap{$\cdots$}, \text{ } \sum\limits_{k=0}^{n-1} \omega^{2(n-1)k}, \text{ } \sum\limits_{k=0}^{n-1} \omega^{2(n+1)k}, \text{ } \gap{$\cdots$}, \text{ } \sum\limits_{k=0}^{n-1} \omega^{2(2n-1)k}$

$ $

\item According to step 1 and 2, the $V \cdot V^T$ matrix has $n$ on its anti-diagonal positions and $0$ for all other positions.

\item Now we will derive the formula for $V^{-1}$. Given $V \cdot V^T = n \cdot I_n^R$, 

$V^{-1} \cdot V \cdot V^T = V^{-1} \cdot n \cdot I_n^R$

$V^T = V^{-1} \cdot n \cdot I_n^R$

$V^T \cdot I_n^R = V^{-1} \cdot n \cdot I_n^R  \cdot I_n^R$

$V^T \cdot I_n^R = V^{-1} \cdot n$ \textcolor{red}{\text{ } \# since $I_n^R  \cdot I_n^R = I_n$}

$ $

Now, there is one caveat: modulo operation does not support direct number division (as explained in \autoref{subsec:modulo-division}). This means that the formula $V^{-1} = \dfrac{V^T \cdot I_n^R}{n}$ in Theorem~\ref*{subsec:vandermonde-euler} (in \autoref{subsec:vandermonde-euler}) is inapplicable in our case, because our modulo $p$ arithmetic does not allow direct division of $V^T \cdot I_n^R$ by $n$. Therefore, we instead multiply $V^T \cdot I_n^R$ by the inverse of $n$ (i.e., $n^{-1}$). We continue as follows:

$V^T \cdot I_n^R = V^{-1} \cdot n$

$V^T \cdot I_n^R \cdot n^{-1}= V^{-1} \cdot n \cdot n^{-1}$

$V^{-1} = n^{-1}\cdot V^T \cdot I_n^R$

\end{enumerate}
\end{proof}

We finally proved that $V\cdot V^T = n \cdot I_n^R$, and $V^{-1} = n^{-1}\cdot V^T \cdot I_n^R$. Later in the BFV scheme (\autoref{sec:bfv}), we will use $V^{-1}$ to encode an integer vector into a vector of polynomial coefficients, and $V^T$ to decode it back to the integer vector (\autoref{subsec:bfv-batch-encoding}).

$ $

\para{Condition for $\bm \mu$:} Like in CKSS, it's worthwhile to note that the property $V\cdot V^T = n\cdot I_n^R$ does not hold if $\mu$ (denoting the $\mu$-th cyclotomic polynomial) is not a power of 2. In particular, step 3 of the proof does not hold anymore if $\mu$ is not a power of 2:

$\sum\limits_{k=0}^{n-1} \omega^{2k} \neq \sum\limits_{k=0}^{n-1} \omega^{4k} \neq \sum\limits_{k=0}^{n-1} \omega^{6k} \neq \gap{$\cdots$} \neq \sum\limits_{k=0}^{n-1} \omega^{2(n-1)k} \neq \sum\limits_{k=0}^{n-1} \omega^{2(n+1)k} \neq \gap{$\cdots$} \neq \sum\limits_{k=0}^{n-1} \omega^{2(2n-1)k} \neq 0$