The Taylor series is a way to represent an analytic function near a point by an infinite power series (assuming the series converges to the function in that neighborhood). Truncating this series gives polynomial approximations of the function. Formally speaking, the Taylor series of a function is an infinite sum of the evaluations of the function's derivatives at a single point. Given function $f(X)$, its Taylor series centered at $a$ is expressed as follows:

$
f(a) + \dfrac{f'(a)}{1!}(X - a) + \dfrac{f''(a)}{2!}(X - a)^2 + \dfrac{f'''(a)}{3!}(X - a)^3 + \cdots = \sum\limits_{d=0}^{\infty}\dfrac{f^{(d)}(a)}{d!}(X - a)^d
$

For a given function, one can truncate the Taylor series to a finite number of terms (up to degree $D$ instead of an infinite number of terms). Such a $D$-degree polynomial is also called the $D$-th Taylor polynomial approximating $f(X)$. Generally, the larger the degree $D$ (i.e. the more terms we include), the more accurate the approximation of $f(X)$ becomes. The accuracy of the approximation is higher for those coordinates nearby $X=a$, and lower for those coordinates away from $X=a$. To increase the accuracy for farther coordinates, we need to increase $D$. 
