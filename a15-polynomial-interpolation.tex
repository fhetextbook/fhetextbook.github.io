Suppose we are given $n+1$ two-dimensional coordinates $(x_0, y_0), (x_1, y_1), \cdots, (x_n, y_n)$, where all $x$ values are distinct, but $y$ values are not necessarily distinct. Lagrange's polynomial interpolation is a technique to find a unique polynomial of degree at most $n$ that passes through such $n+1$ coordinates. The given points $(x_i,y_i)$ may lie either in $\mathbb{C}^2$ (the complex plane, which includes the real numbers) or in $\mathbb{Z}_p^2$ for some prime $p$.


\begin{tcolorbox}[title={\textbf{\tboxtheorem{\ref*{sec:polynomial-interpolation}} Lagrange's Polynomial Interpolation}}]

Suppose we are given $n+1$ two-dimensional coordinates $(x_0, y_0), (x_1, y_1), \cdots, (x_n, y_n)$, whereas all $X$ values are distinct but the $Y$ values don't need to be distinct. The domain of $(X, Y)$ can be either: $(x_i, y_i) \in \mathbb{C}^2$ (which includes the real domain) or $(x_i, y_i) \in \mathbb{Z}_p^2$ (where $p$ is a prime). Then, there exists a unique polynomial $f(X)$ of degree at most $n$ that passes through these $n+1$ coordinates. Such a polynomial $f(X)$ is computed as follows:

$f(X) = \sum\limits_{j=0}^{n}\dfrac{(X-x_0)\cdot(X-x_1)\cdots(X-x_{j-1})\cdot(X-x_{j+1})\cdots(X-x_{n})}{(x_j-x_0)\cdot(x_j-x_1)\cdots(x_j-x_{j-1})\cdot(x_j-x_{j+1})\cdots(x_j-x_{n})}\cdot y_j$

$\textcolor{white}{f(X) }= \sum\limits_{j=0}^{n} \left( \prod\limits_{\substack{0 \le k \le n\\ k \ne j}} \dfrac{X - x_k}{x_j - x_k} \cdot y_j\right)$

\end{tcolorbox}

\begin{myproof}
\begin{enumerate}


\item First, we will show that there exists an $n$-degree (or lesser degree) polynomial $f(X)$ that passes through the $n+1$ distinct coordinates: $(x_0, y_0), (x_1, y_1), \ldots, (x_n, y_n)$. Such a polynomial $f(X)$ is designed as follows:

$f(X) = \sum\limits_{j=0}^{n}\dfrac{(X-x_0)\cdot(X-x_1)\cdots(X-x_{j-1})\cdot(X-x_{j+1})\cdots(X-x_{n})}{(x_j-x_0)\cdot(x_j-x_1)\cdots(x_j-x_{j-1})\cdot(x_j-x_{j+1})\cdots(x_j-x_{n})}\cdot y_j$

$ \textcolor{white}{f(X) }= \sum\limits_{j=0}^{n} \left( \prod\limits_{\substack{0 \le k \le n\\ k \ne j}} \dfrac{X - x_k}{x_j - x_k} \cdot y_j\right)$

$ \textcolor{white}{f(X) }= \sum\limits_{j=0}^{n} \ell_j(X) \cdot y_j$ \textcolor{red}{ $\rhd$ where $\ell_j(X) =  \prod\limits_{\substack{0 \le k \le n\\ k \ne j}} \dfrac{X - x_k}{x_j - x_k}$}

$ $

We call $\{\ell_0(X), \ell_1(X), \ldots, \ell_{n}(X) \}$ the Lagrange basis for polynomials of degree $\leq n$. Given this design of $f(X)$, notice that for each of $(x_i, y_i) \in \{(x_0, y_0), (x_1, y_1), \ldots, (x_n, y_n)\}$, \text{ } $\ell_i(x_{i'}) = 1$ for $i' = i$, and $\ell_i(x_{i'}) = 0$ for $i' \neq i$. Therefore, $f(x_i) = \sum\limits_{j=0}^{n} \ell_j(x_i) \cdot y_j = 1 \cdot y_i = y_i$ for $0 \leq i \leq n$. In other words, $f(X)$ passes through the $n+1$ distinct coordinates: $\{(x_0, y_0), (x_1, y_1), \ldots, (x_n, y_n)\}$. Such a satisfactory $f(X)$ can be computed in the case where the domain of $(X, Y$) is either: $(x_i, y_i) \in \mathbb{C}^2$ (i.e., real and complex numbers), or $(x_i, y_i) \in \mathbb{Z}_p^2$ (where $p$ is a prime). Especially, a valid $f(X)$ can be computed also in the $\bmod \, p$ domain, because as we learned from Fermat's Little Theorem in Theorem~\ref*{subsec:order-theorem}.4 (\autoref{subsec:order-theorem}), $a^{p - 1} \equiv 1 \bmod p$ if and only if $a$ and $p$ are co-prime, and this means that if $p$ is a prime, then $a^{p - 1} \equiv 1 \bmod p$ for all $a \in \mathbb{Z}_p^{\times}$ (i.e., $\mathbb{Z}_p$ without $\{0\}$). Since every value in $\mathbb{Z}_p^{\times}$ has an inverse, 
we can perform each division in the formula for $f(X) = \sum\limits_{j=0}^{n}\dfrac{(X-x_0)\cdot(X-x_1)\cdots(X-x_{j-1})\cdot(X-x_{j+1})\cdots(X-x_{n})}{(x_j-x_0)\cdot(x_j-x_1)\cdots(x_j-x_{j-1})\cdot(x_j-x_{j+1})\cdots(x_j-x_{n})}\cdot y_j$ by multiplying with the corresponding inverses of those denominators.
 
\item Next, we will prove that no two distinct $n$-degree (or lesser degree) polynomials $f_1(X)$ and $f_2(X)$ can pass through the same $n+1$ distinct $(X, Y)$ coordinates. Suppose there exist such two polynomials $f_1(X)$ and $f_2(X)$. Then $f_{1-2}(X) = f_1(X) - f_2(X)$ will be a new $n$-degree (or lesser degree) polynomial that passes through $(x_0, 0), (x_1, 0), \cdots, (x_n, 0)$. This means that $f_{1-2}(X)$ has $n+1$ distinct roots. In other words, $f_{1-2}(X)$ is an $n+1$-degree (or higher-degree) polynomial. However, this contradicts the assumption that $f_{1-2}(X)$ is an $n$-degree (or lesser degree) polynomial. Therefore, there exist no two polynomials $f_1(X)$ and $f_2(X)$ that pass through the same $n+1$ distinct $(X, Y)$ coordinates. 

\item We have shown that there exists some $n$-degree (or lesser degree) polynomial $f(X)$ that passes through $n+1$ distinct $(X, Y)$ coordinates, and no such two or more distinct polynomials exists. Therefore, there exists only a unique polynomial that satisfies this requirement. 

\end{enumerate}
\end{myproof}

