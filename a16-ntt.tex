\textbf{- Reference:} 
\href{http://web.cecs.pdx.edu/~maier/cs584/Lectures/lect07b-11-MG.pdf}{Polynomials and the Fast Fourier Transform (FFT)}~\cite{ntt}

\subsection{Background and Motivation}
\label{subsec:ntt-motivation}

Given two $(n-1)$-degree polynomials:

$A(X) = \sum\limits_{i=0}^{n-1}a_iX^i$, \textcolor{white}{...} $B(X) = \sum\limits_{i=0}^{n-1}b_iX^i$

, the polynomial multiplication $C(X) = A(X)\cdot B(X)$ is computed as follows:

$C(X) = \sum\limits_{i=0}^{2n-1}c_iX^{i}$, where $c_i = \sum\limits_{k=0}^{i}a_kb_{i-k}$

This operation of computing $\vec{c} = (c_0, c_1, \cdots, c_{2n-1})$ is also called the convolution of $\vec a$ and $\vec b$, denoted as $\vec{c} = \vec{a} \otimes \vec{b}$. The time complexity of this operation (i.e., the total number of multiplications between two numbers) is $O(n^2)$. 

Another way of multiplying two polynomials is based on \textbf{point-value representation}. The point-value representation of an $(n-1)$-degree (or lesser degree) polynomial $A(X)$ is a set of $n$ coordinates $\{(x_0, y_0), (x_1, y_1), \cdots (x_{n-1}, y_{n-1})\}$, where each $x_i$ is a distinct $X$ coordinate (whereas each $y_i$ is not necessarily a distinct $Y$ coordinate). Given a point-value representation of an $(n-1)$-degree (or lesser degree) polynomial, we can use polynomial interpolation (\autoref{sec:polynomial-interpolation}) to derive the polynomial. 
Let's denote the point-value representation of $(n-1)$-degree (or lesser degree) polynomial $A(X)$ and $B(X)$ as follows:

$A(X)$ : $\bm ( ({x}_0, {y}_0^{\langle a \rangle}), ({x}_1, {y}_1^{\langle a \rangle}), \cdots ({x}_{n-1}, {y}_{n-1}^{\langle a \rangle}) \bm )$

$B(X)$ : $\bm ( ({x}_0, {y}_0^{\langle b \rangle}), ({x}_1, {y}_1^{\langle b \rangle}), \cdots ({x}_{n-1}, {y}_{n-1}^{\langle b \rangle}) \bm )$


Then, the point-value representation of the polynomial $C(X) = A(X) \cdot B(X)$ can be computed as a Hadamard product (Definition~\ref*{subsec:vector-arithmetic} in \autoref{subsec:vector-arithmetic}) of the $y$ values of the point-value representation of $A(X)$ and $B(X)$ as follows:

$C(X)$ : $\bm ( ({x}_0, {y}_0^{\langle c \rangle}), ({x}_1, {y}_1^{\langle c \rangle}), \cdots ({x}_{n-1}, {y}_{n-1}^{\langle c \rangle}) \bm )$, where ${y}_i^{\langle c \rangle} = {y}_i^{\langle a \rangle} \cdot {y}_i^{\langle b \rangle}$

However, we cannot derive polynomial $C(X)$ based on these $n$ coordinates because the degree of $C(X)$ is $2n-2$ (or less than $2n-2$). But if we regard all polynomials (including $A(X), B(X)$ and $C(X)$) to be in the polynomial ring $\mathbb{R}[X]/ X^n + 1$ (or $\mathbb{Z}_p[X]/ X^n + 1$), then we can reduce the $(2n-2)$-degree polynomial $C(X)$ to a congruent $(n-1)$-degree (or lesser degree) polynomial in the ring. Then, the $n-1$ coordinates of $C(X)$ are sufficient to derive $C(X)$. 

However, the time complexity of this new method is still $O(n^2)$. The Hadamard product between two polynomials' point-value representations takes $O(n)$, but evaluating a polynomial at $n$ distinct $x$ values takes $O(n^2)$ (because each polynomial has $n$ terms, and we have to compute each term for $n$ distinct $x$ values). The polynomial interpolation for deriving $C(X)$ also takes $O(n^2)$. 

To solve this efficiency problem, this section will introduce an efficient technique for polynomial evaluation, which can evaluate a polynomial at $n$ distinct roots of unity in $O(n \log n)$. This technique is classified into 2 types: Fast Fourier Transform (FFT) and Number-theoretic Transform(NTT). These two types are technically almost the same, with the only difference being that the FFT assumes a polynomial ring over complex numbers (\autoref{sec:roots}), whereas the NTT assumes a polynomial ring over the integers (\autoref{sec:cyclotomic-polynomial-integer-ring}). Polynomial multiplication based on FFT (or NTT) comprises 3 steps: (1) forward FFT (or NTT); (2) point-value multiplication; and (3) inverse FFT (or NTT). 

\subsection{Forward FFT (or NTT)}
\label{subsec:ntt-forward}

We assume a polynomial ring of $\mathbb{R}[X]/ X^n + 1$ for FTT, and $\mathbb{Z}_p[X]/ X^n + 1$ for NTT (where $X^n + 1$ is a cyclotomic polynomial). The $x$ coordinates to evaluate the target polynomial are the $n$ distinct roots of $X^n + 1$, which are $\omega^1$, $\omega^3, \ldots, \omega^{2n-1}$, where $\omega$ is the primitive $2n$-th root of unity. Then, the point-value representation of the polynomial $A(X)$ is $\bm ( ({x}_0, {y}_0^{\langle a \rangle}), ({x}_1, {y}_1^{\langle a \rangle}), \cdots ({x}_{n-1}, {y}_{n-1}^{\langle a \rangle}) \bm )$, where: 

${y}_i^{\langle a \rangle} = A(\omega^{2i + 1}) = \sum\limits_{j=0}^{n-1}a_{j}\cdot (\omega^{2i + 1})^j = \sum\limits_{j=0}^{n-1}a_{j}\cdot \omega^{(2i + 1)\cdot j}$. 

We call the vector $\vec{y}^{\langle a \rangle} = (y_0^{\langle a \rangle}, y_1^{\langle a \rangle}, \cdots, y_{n-1}^{\langle a \rangle})$ the Discrete Fourier Transform (DFT) of the coefficient vector $\vec{a} = (a_0, a_1, \cdots, a_{n-1})$. We write this as $\vec{y}^{\langle a \rangle} = \textsf{DFT}(\vec{a})$. As explained in \autoref{subsec:ntt-motivation}, the computation of the DFT takes $O(n^2)$, because we have to evaluate $n$ distinct $X$ values for a polynomial that has $n$ terms.

\subsubsection{High-level Idea}
\label{subsec:ntt-forward-overview}

FFT (or NTT) is an improved method for computing the DFT, which reduces the time complexity from $O(n^2)$ to $O(n \log n)$. The high-level idea of FFT is to split the $(n-1)$-degree (or lesser degree) target polynomial $A(X)$ to evaluate into 2 half-degree polynomials $A_0(X)$ and $A_1(X)$ as follows:


$A(X) = a_0 + a_1X + a_2X^2 + \cdots + a_{n-1}X^{n - 1}$

$\textcolor{white}{A(X) }= A_0(X^2) + X \cdot A_1(X^2)$
$A_0(X) = a_0 + a_2X + a_4X^2 + \cdots + a_{n-2}X^{\frac{n}{2} - 1}$

$A_1(X) = a_1 + a_3X + a_5X^2 + \cdots + a_{n-2}X^{\frac{n}{1} - 1}$

The above method of splitting a polynomial into two half-degree polynomials is called the Cooley-Tukey step. As we split $A(X)$ into two smaller-degree polynomials $A_0(X)$ and $A_1(X)$, evaluating $A(X)$ at the odd-powered primitive $2n$-th roots of unity $\{\omega^1, \omega^3, \omega^5, \cdots, \omega^{2n-1}\}$ is equivalent to evaluating $A_0(X)$ and $A_1(X)$ at $n$ distinct \textit{squared} $n$-th roots of unity $\{(\omega^2)^1, (\omega^2)^3, (\omega^2)^5, \ldots, (\omega^2)^{2n-1}\}$ and computing $A_0(X^2) + X\cdot A_1(X^2)$. However, remember that the primitive $2n$-th root of unity $\omega$ has order $2n$ (i.e., $\omega^{2n} = 1$ and $\omega^m \neq 1$ for all $m < 2n$). Therefore, the second half of $\{(\omega^2)^1, (\omega^2)^3, (\omega^2)^5, \ldots, (\omega^2)^{2n-1}\}$ is a repetition of the first half. This implies that we only need to evaluate $A_0(X)$ and $A_1(X)$ at $\dfrac{n}{2}$ distinct $x$ coordinates each, instead of $n$ distinct coordinates, because the polynomial evaluation results for the other half are the same as those of the first half (as their input $x$ to the polynomial is the same). 

We recursively split $A_0(X)$ and $A_1(X)$ into half-degree polynomials and evaluate them at half-counted $n$-th roots of unity. Then, the total number of rounds of splitting is $\log n$, and the maximum number of root-to-coefficient multiplications in each round is $n$, which aggregates to $O(n \log n)$. 

\subsubsection{Details}
\label{subsec:ntt-forward-details}

Suppose we have a polynomial ring that is either $\mathbb{Z}_{p}[X] / X^8 + 1$ (i.e., over a finite field with prime $p$) or $\mathbb{R}[X] / X^8 + 1$ (over complex numbers). 
%$g = 3$ is the generator of $\mathbb{Z}_{17}^* = \{1, 2, \cdots, 16\}$. 
We denote the primitive $(2n=8)$-th roots of unity as $\omega$, and 
%which is $\omega = e^{\frac{i 2\pi}{n}}$ and %$ = g^{\frac{p - 1}{n}}$.
%= 3^{\frac{17 - 1}{8}} = 9$. 
the $n$ distinct $(n=8)$-th roots of $X^n + 1$ are: $\{\omega^1, \omega^3, \omega^5, \omega^7, \omega^9, \omega^11, \omega^13, \omega^15\}$.
%$ \bmod 17 \equiv \{1, 9, 13, 15, 16, 8, 4, 2\} \bmod 17$. 

$ $

Now, we define our target polynomial to evaluate as follows: 

$A(X) = a_0 + a_1X + a_2X^2 + a_3X^3 + a_4X^4 + a_5X^5 + a_6X^6 + a_7X^7$

$ $

We split this 7-degree polynomial into the following two 3-degree polynomials (using the Cooley-Tukey step): 

$A_0(X) = a_0 + a_2X + a_4X^2 + a_6X^3$

$A_1(X) = a_1 + a_3X + a_5X^2 + a_7X^3$

$A(X) = A_0(X^2) + X \cdot A_1(X^2)$

$ $

We recursively split the two 3-degree polynomials above into 1-degree polynomials as follows:


$A_{0,0}(X) = a_0 + a_4X$, \textcolor{white}{...} $A_{0,1}(X) = a_2 + a_6X$

$A_0(X) = A_{0,0}(X^2) + X\cdot A_{0,1}(X^2)$

$ $

$A_{1,0}(X) = a_1 + a_5X$, \textcolor{white}{...} $A_{1,1}(X) = a_3 + a_7X$

$A_1(X) = A_{1,0}(X^2) + X\cdot A_{1,1}(X^2)$

$ $

$A(X) = A_0(X^2) + X \cdot A_1(X^2)$

$\mathcolor{white}{A(X)} = \underbrace{\underbrace{(\underbrace{A_{0,0}(X^4)}_{\text{FFT Level 1}} + X^2\cdot \underbrace{A_{0,1}(X^4)}_{\text{FFT Level 1}})}_{\text{FFT Level 2}} + X \cdot \underbrace{(\underbrace{A_{1,0}(X^4)}_{\text{FFT Level 1}} + X^2\cdot \underbrace{A_{1,1}(X^4)}_{\text{FFT Level 1}})}_{\text{FFT Level 2}}}_{\text{FFT Level 3}}$

$ $

To evaluate $A(X)$ at the $n$ distinct roots $X = \{\omega^1, \omega^3, \ldots, \omega^15\}$, we evaluate each FFT level of the above formula at $X = \{\omega^1, \omega^3, \cdots, \omega^15\}$, starting from level $1 \leq l \leq 3$. 

$ $

\para{FFT Level $\bm{l = 1}$:} We evaluate $A_{0,0}(X^4)$, $A_{0,1}(X^4)$, $A_{1,0}(X^4)$, and $A_{1,1}(X^4)$ at $X = \{\omega^1, \omega^3, \cdots, \omega^15\}$. However, notice that plugging in $X = \{\omega^1, \omega^3, \cdots, \omega^15\}$ to $X^4$ results in only 2 distinct values: $\omega^0$ and $\omega^8$. This is because the order of $\omega$ is $2n$ (i.e., $\omega^{2n} = 1$), and thus $(\omega^1)^4 = (\omega^5)^4 = (\omega^9)^4 = (\omega^{13})^4$, and $(\omega^3)^4 = (\omega^7)^4 = (\omega^{11})^4 = (\omega^{15})^4$. Therefore, we only need to evaluate $A_{0,0}(X^4)$, $A_{0,1}(X^4)$, $A_{1,0}(X^4)$, and $A_{1,1}(X^4)$ at 2 distinct $x$ values instead of 8, where each evaluation requires a constant number of arithmetic operations: computing 1 multiplication and 1 addition. As there are a total of 4 polynomials to evaluate (i.e., $A_{0, 1}(X^4), A_{0, 1}(X^4), A_{1, 0}(X^4), A_{1, 1}(X^4)$), we compute the FFT a total of $4 \cdot 2 = 8$ times. 

$ $

\para{FFT Level $\bm{l = 2}$:} Based on the evaluation results from FFT Level 1 as building blocks, we evaluate $A_{0}(X^2)$ and $A_{1}(X^2)$ at $X = \{\omega^1, \omega^3, \cdots, \omega^{15}\}$. However, notice that plugging in $X = \{\omega^1, \omega^3, \cdots, \omega^15\}$ to $X^2$ results in only 4 distinct values: $\omega^1$, $\omega^5$, $\omega^9$, and $\omega^{13}$.
This is because the order of $\omega$ is $n$ (i.e., $\omega^n = 1$), and thus $(\omega^1)^2 = (\omega^9)^2$, $(\omega^3)^2 = (\omega^{11})^2$, $(\omega^5)^2 = (\omega^{13})^2$, and $(\omega^7)^2 = (\omega^{15})^2$. Therefore, we only need to evaluate $A_{0}(X^2)$ and $A_{1}(X^2)$ at 4 distinct $x$ values instead of 8, where each evaluation requires a constant number of arithmetic operations: computing 1 multiplication and 1 addition (where we use the results from FFT Level 1 as building blocks, and the computational structure of FFT Level 2 is the same as that of FFT Level 1). There are a total of 2 polynomials to evaluate (i.e., $A_{0}(X^2), A_{1}(X^2)$); thus, we compute the FFT a total of $2 \cdot 4 = 8$ times. 

$ $

\para{FFT Level $\bm{l = 3}$:} Based on the evaluation results from FFT Level 2 as building blocks, we evaluate $A(X)$ at $X = \{\omega^1, \omega^3, \cdots, \omega^{15}\}$. For this last level of computation, we need to evaluate all 8 distinct $X$ values, since they are all unique values, and each evaluation requires a constant number of arithmetic operations: computing 1 multiplication and 1 addition. There is a total of 1 polynomial to evaluate (i.e., $A(X)$); thus, we compute the FFT a total of $1 \cdot 8 = 8$ times. 

$ $

\para{Generalization:} Suppose that the degree of the target polynomial to evaluate is bounded by $n$ degree, and we define $L = \log n$ (i.e., the total number of FFT levels). Then, the forward FFT operation requires a total of $L$ FFT levels, where each $l$-th level requires the evaluation of $2^{L - l}$ polynomials at $2^l$ distinct $X$ values. Therefore, the total number of FFT computations for the forward FFT is: $\log (n) \cdot (2^{L - l} \cdot 2^l) = 2^L \log n = n \log n$. Therefore, the time complexity of the forward FFT is $O(n \log n)$.

Using the FFT technique, we reduce the number of $x$ points to evaluate to half as the level decreases (while the number of polynomials to evaluate doubles), and their growth and reduction cancel each other, resulting in $O(n)$ for each level. Since there are $\log n$ such levels, the total time complexity is $O(n \log n)$. The core enabler of this optimization is the special property of the $x$ evaluation coordinates: its power (i.e., $\omega^i$) is cyclic. To enforce this cyclic property, FFT requires the evaluation points of $x$ to be the odd-powered primitive $2n$-th roots of unity. 


\subsection{Point-wise Multiplication}
\label{subsec:pointwise-multiplication}

Once we have applied the forward FFT operation (\autoref{subsec:ntt-forward}) to polynomial $A(X)$ and $B(X)$ as $\vec{y}^{\langle a \rangle}$ and $\vec{y}^{\langle b \rangle}$, computing the point-value representation of $C(X) = A(X) \cdot B(X)$ can be done in $O(\log n)$ using the Hadamard product $ \vec{y}^{\langle c \rangle} = \vec{y}^{\langle a \rangle} \odot \vec{y}^{\langle b \rangle}$ (as explained in \autoref{subsec:ntt-motivation}).

\subsection{Inverse FFT (or NTT)}
\label{subsec:ntt-backward}

Once we have computed:

$C(X)$ : $\bm ( ({x}_0, {y}_0^{\langle c \rangle}), ({x}_1, {y}_1^{\langle c \rangle}), \cdots ({x}_{n-1}, {y}_{n-1}^{\langle c \rangle}) \bm )$, where ${y}_i^{\langle c \rangle} = {y}_i^{\langle a \rangle} \cdot {y}_i^{\langle b \rangle}$

, our final step is to convert ${y}_i^{\langle c \rangle}$ back to $\vec{c}$, the polynomial coefficients of $C(X)$. We call this reversing operation the inverse FFT. 

Given an $(n-1)$-degree polynomial 
$A(X) = \sum\limits_{i=0}^{n-1}a_iX^i$, the forward FFT process is computationally equivalent to evaluating the polynomial at $n$ distinct $n$-th roots of unity as follows:

${y}_i^{\langle a \rangle} = A(\omega^i) = \sum\limits_{j=0}^{n-1}a_{j}\cdot (\omega^i)^j = \sum\limits_{j=0}^{n-1}a_{j}\cdot \omega^{ij}$

The above evaluation is equivalent to computing the following matrix-vector multiplication: 

$ $

${y}_i^{\langle a \rangle} = W \cdot \vec{a}$, \text{ } where $W = \begin{bmatrix}
(\omega^0)^0 & (\omega^0)^1 & \cdots & (\omega^0)^{n-1}\\
(\omega^1)^0 & (\omega^1)^1 & \cdots & (\omega^1)^{n-1}\\
\vdots & \vdots & \ddots & \vdots \\
(\omega^{n-1})^0 & (\omega^{n-1})^1 & \cdots & (\omega^{n-1})^{n-1}\\
\end{bmatrix} $
, \text{ } $\vec{a} = (a_0, a_1, \cdots, a_{n-1})$


$ $

We denote each element of $W$ as: $(W)_{i,j} = \omega^{ij}$. The inverse FFT is a process of reversing the above computation. For this inversion, our goal is to find an inverse matrix $W^{-1}$ such that $W^{-1} \cdot {y}_i^{\langle a \rangle} = W^{-1} \cdot (W \cdot \vec{a}) = (W^{-1} \cdot W) \cdot \vec{a} = I_n \cdot \vec{a} =  \vec{a}$. As a solution, we propose the inverse matrix $W^{-1}$ as follows: $(W^{-1})_{j,k} = n^{-1}\cdot \omega^{-jk}$. Now, we will show why $W^{-1} \cdot W = I_n$. 

Each element of $W^{-1} \cdot W$ is computed as: 

$(W^{-1} \cdot W)_{j, i} = \sum\limits_{k=0}^{n-1}(n^{-1}\cdot \omega^{-jk} \cdot \omega^{ki}) = n^{-1}\cdot\sum\limits_{k=0}^{n-1}\omega^{-jk} \cdot \omega^{ki} = n^{-1}\cdot\sum\limits_{k=0}^{n-1}\omega^{(i-j)\cdot k}$

In order for $W^{-1} \cdot W$ to be $I_n$, the following should hold:

\[
(W^{-1} \cdot W)_{j, i} = \begin{cases}
\text{1 if } j = i \\
\text{0 if } j \neq i \\
\end{cases}
\]

If $j = i$, the above condition holds, because $(W^{-1} \cdot W)_{j, i} = n^{-1}\cdot\sum\limits_{k=0}^{n-1}\omega^{(0)\cdot k} =  n^{-1}\cdot\sum\limits_{k=0}^{n-1}1 = n^{-1} \cdot n = 1$. 

In the case of $j \neq i$, we will leverage the Geometric Sum formula $\sum\limits_{i=0}^{n-1}x^i = \dfrac{x^n - 1}{x - 1}$ (the proof is provided below):


\begin{tcolorbox}[title={\textbf{\tboxtheorem{\ref*{subsec:ntt-backward}} Geometric Sum Formula}}]

Let the geometric sum $S_n = 1 + x + x^2 + \cdots + x^{n - 1}$

Then, $x \cdot S_n = x + x^2 + x^3 + \cdots + x^{n}$

$x \cdot S_n - S_n = (x + x^2 + x^3 + \cdots + x^{n}) - (1 + x + x^2 + \cdots + x^{n - 1}) = x^n - 1$

$S_n\cdot (x - 1) = x^n - 1$

$S_n = \dfrac{x^n - 1}{x - 1}$ \textcolor{red}{ \# with the constraint that $x \neq 1$}

\end{tcolorbox}

Our goal is to compute $\sum\limits_{k=0}^{n-1}\omega^{(i-j)\cdot k}$. Leveraging the Geometric Sum formula $\sum\limits_{i=0}^{n-1}x^i = \dfrac{x^n - 1}{x - 1}$, 

$\sum\limits_{k=0}^{n-1}\omega^{(i-j)\cdot k} = \dfrac{(\omega^{i - j})^n - 1}{\omega^{i - j} - 1}$ 

$= \dfrac{(\omega^n)^{i - j} - 1}{\omega^{i - j} - 1}$

$= \dfrac{(1)^{i - j} - 1}{\omega^{i - j} - 1}$ \textcolor{red}{ \# since the order of $\omega$ is $n$}

\textcolor{red}{ \# Here, the denominator can't be 0, since $i \neq j$ and $-n < i - j < n$ (as $0 \leq i < n$ and $0 \leq j < n$)}

$= 0$

Thus, $(W^{-1} \cdot W)_{j, i}$ is 1 if $j = i$, and 0 if $j \neq i$. Therefore, the inverse FFT can be computed as follows:


$\vec{c} = W^{-1} \cdot y_i^{\langle c \rangle}$, \text{ } where $c_i = n^{-1}\cdot \sum\limits_{j=0}^{n-1}{y}_i^{\langle c \rangle} \cdot \omega^{-ij}$

The above inverse FFT computation is equivalent to evaluating the polynomial $\hat{C}(X) = \sum\limits_{i=0}^{n-1} y_i^{\langle c \rangle} \cdot X^i$ at $X = \{\omega^{0}, \omega^{-1}, \omega^{-2}, \cdots,  \omega^{-(n-1)}\}$, which is equivalent to evaluating $\hat{C}(X)$ at $X = \{\omega^{0}, \omega^{n-1}, \omega^{n-2}, \cdots, \omega^1\}$. For this evaluation, we can use the same optimization technique explained in the forward FFT process (\autoref{subsec:ntt-forward-details}) whose time complexity is $O(n \log n)$.

\begin{comment}
\begin{algorithm}[t]
\algrenewcommand\algorithmicindent{1em} % Adjust indentation herei.e., 
\begin{algorithmic}[1]
\State \textsf{FFT}$(\vec{a}):$ %\textcolor{red}{\# an RLWE ciphertext storing $k$ zeros and non-zero random numbers, whose ratio varies}
\State $n' \gets$ length of $\vec{a}$
\If{$n' = 1$}
    \State \textbf{return} $a$
\EndIf
\State $\omega = e^{\frac{2\pi}{k}}$ \textcolor{red}{ \# or $\omega = g^{\frac{p-1}{n}}$ in the case of NTT}
\State $\omega' = 1$
\State $\vec{a}^{\langle 0 \rangle} = (a_0, a_2, \cdots, a_{n'-2})$
\State $\vec{a}^{\langle 1 \rangle} = (a_1, a_3, \cdots, a_{n'-1})$
\State $\vec{y}^{\langle 0 \rangle} = \textsf{FFT}(\vec{a}^{\langle 0 \rangle})$
\State $\vec{y}^{\langle 1 \rangle} = \textsf{FFT}(\vec{a}^{\langle 1 \rangle})$
\For{$i \in \{0, 1, \cdots \frac{n'}{2} - 1\}$}
    \State $y_i \gets y_i^{\langle 0 \rangle} + \omega \cdot y_i^{\langle 1 \rangle}$
    \State $y_{i+\frac{n}{2}} \gets y_i^{\langle 0 \rangle} + \omega \cdot y_i^{\langle 1 \rangle}$
    \State $\omega' = \omega' \cdot \omega$
\EndFor
\State \textbf{return} $\vec{y}$ 
\end{algorithmic}
\caption{Fast Fourier Transform (FFT)}
\label{alg:fft}
\end{algorithm}
\end{comment}