A GLev ciphertext is a list of GLWE ciphertexts that encrypt the list of plaintexts $\dfrac{q}{\beta^1}M, \dfrac{q}{\beta^2}M, \ldots, \dfrac{q}{\beta^l}M$, where $M$ is a plaintext encoded in a polynomial. Note that each $i$-th GLWE ciphertext of a GLev ciphertext uses a different plaintext scaling factor, which is: $\Delta_i = \dfrac{q}{\beta^i}$. The structure of GLev ciphertext is visually depicted in \autoref{fig:glev}.


Note that $\beta$ should be some value between $t$ and $q$. Specifically, $t$ should be smaller than or equal to $\beta$ because if $t$ is greater than $\beta$, then the higher bits of $M$ will overflow beyond $q$ when computing $\dfrac{q}{\beta^1}M$. 


\subsection{Encryption}
\label{subsec:glev-enc}

\begin{tcolorbox}[title={\textbf{\tboxlabel{\ref*{subsec:glev-enc}} GLev Encryption}}]


$\textsf{GLev}_{S, \sigma}^{\beta, l}(M) = \Bigl \{\textsf{GLWE}_{S, \sigma}\left(\dfrac{q}{\beta^i} M + E_i\right)  \Bigr \}_{i=1}^{l} \in \mathcal{R}_{\langle n, q \rangle }^{(k+1) \cdot l}$
\end{tcolorbox}

\begin{figure}[h!]
    \centering
  \includegraphics[width=1.0\linewidth]{figures/TFHE-fig2.pdf}
  \caption{An illustration of a GLev ciphertext }
  \label{fig:glev}
\end{figure}



\subsection{Decryption}

We decrypt the first GLWE ciphertext ($i=1$) using the secret $S$, with the scaling factor $\Delta_1 = \dfrac{q}{\beta}$. This is because while the ciphertext contains $l$ encryptions, the higher indices $i > 1$ have progressively smaller scaling factors $\Delta_i = q/\beta^i$. If $\Delta_i$ becomes smaller than the noise threshold, those specific components cannot be decrypted correctly.

\subsection{Lev and RLev}

Lev is GLev with $n=1$. RLev is GLev with $k=1$.

